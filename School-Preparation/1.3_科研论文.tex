
\section{科研论文}
科研更像是一种探索客观世界发展规律的方法论。很多时候并没有所谓的正确答案。当你不断质疑他人的方法论并反复验证以后,你就能找到属于自己的且更长久的科研秘籍。形成科研训练是一个非常漫长的过程,需要长时间付出。首先你需要阅读大量论文,形成行业的前沿发展认知。其次你需要拥有创新思维,并在代码中付出实践,然后你需要反复实验证明结论的可靠性,最后需要用PPT的形式向他人通俗易懂的证明你的方法深邃性。


1)阅读论文。建议看英文论文。在培养行业思维的同时,可以看到别人是怎么创造单词的,比如为什么命名成ResNet,大家惯用的命名方式是哪些,如何可以通过名字就能让人一眼就知道你的方法的核心卖点。

2)大量阅读文献。大量的文献阅读可以让你慢慢的感受到不同人对同一个方法论的观点的批判,让你获得不同的视角语言。举个例子,我把所有层全部连了起来,于是有了DesNet(并不严谨),全连起来效果肯定更好啊,性能提升很多。你会觉得那都那么完美了,还有什么可做的。但可能下一篇文章会告诉你这样会存在大量冗余线路,全连起来并不是一个有效的路径依赖关系,当数据量不够大的时候,更容易存在过拟合现象。大量的文献阅读让你不断的被迫的学会要反驳前人的观点,没有一个方法是完美的,前人也存在自身的局限性。

3)培养创新思维。 当你大量阅读完文献,你就需要想想还有什么可改进的。只要你读的够多,上述的思维不断重复出现,很自然就会有一些创造性的价值点,所以这是一个过程,如果这个时候导师对你进度并不满意,建议问问同领域的师兄师姐,看看你的idea到底哪里考虑的不清楚。

4)代码实践科研。请你大量code练习,如果你只想硕士毕业,那code能力是你能独立做完完整科研论文的最重要环节。多关注自身的编程能力,多花时间实践,一些小技巧就是,别把时间花在不重要的模块上(这个要看个人的感觉,到底哪里不重要)。 

5)实验验证效果。大部分情况,你的方法并不能在所有场景下都效果奇佳,多考虑为什么效果不好。不用太在意自己的方法,透过方法,看到事物的本质,没准能在另一个场景下获得出其不意的效果。比如,之前有个工作,一直纠结于如何替换更好的语言模型,后来我发现对于这个场景,如何更好的定位目标比靠语言模型有用太多,不能钻牛角尖。 

6)形成学术论文。虽然写到最后一个环节,但对于大部分对科研训练陌生的朋友来说,这个时候可能才刚刚开始。如何形成属于自己的科研卖点,怎么写intro,ab,这些可能都需要反复斟酌。大部分审稿人第一印象在于你的ab,而看你故事讲的好不好在你的intro。

做科研最直观的价值在于形成系统的科研学术思维。论文算是申请博士的硬通货,当你以后励志于读博,科研论文将对你有很大的帮助。当你的博士导师看过并赞叹过你的科研方法时,哪怕你只有1-2篇paper,你也能申请到MIT级别的CS博士(真事),请不要太在意数量,更多的在意质量。在没有超高质量的前提下,申请的科研关联性也是博士生导师很在意的方向之一。众所周知,计算机的科研论文不仅在学术道路上占有很重要的价值,在算法大厂工作申请中也有一定的作用。只是因为AI领域CCF-A类论文数量的井喷式发展,AI的大厂对论文的在意程度下降明显。不过如果是网络,安全,数据库等领域的A类高含金量论文,在大厂里仍然有很高的认可度。用之前leader的一句话讲这个逻辑就是:你能从一堆做科研的人里脱引而出,发表高含金量(物以稀为贵)的论文,那就是证明你聪明/优秀,大厂需要这样的人。所以如果你有这两点的诉求,请好好的进行你的科研工作。