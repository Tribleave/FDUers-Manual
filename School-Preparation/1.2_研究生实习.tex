\section{研究生实习}
关于如何为简历、笔试、面试准备,在章节2.1.3中有详细阐述,此处不详细展开,就讲一些笔试和面试准备以外的东西。

\subsection{实习的目的}
实习的目的,主要有以下两点:

\textbf{简历加分:}对开发岗来说,实习经历至关重要,实习经历大于校内任何项目。对于算法岗来说同样如此,如果你做的是业务算法(目前硕士基本都是业务算法),一段对口的大厂算法实习经历,含金量近似于一篇对口方向的顶会一作,更胜过不对口方向的顶会一作。条件允许的情况下,尽可能多地进行实习,给自己的简历加分。

\textbf{工作保底:}暑期实习是获得正式offer相对最快、难度最小的方式,可以作为秋招时的保底选项,能够帮助argue薪酬和稳住心态。

\subsection{实习机会}
获得实习的方式,主要有以下三种:

\textbf{官网投递:}这是寻找实习的最主要方式,从各大公司的招聘官网进行投递。官网的信息会比boss、牛客等招聘网站更新更加及时,有时会有一些急招的岗位放出,通过的概率更大。此外,有事没事可以更新一下自己的简历,更新简历后,简历会出现在人才库的前面,更容易被面试官看到和捞面。

\textbf{内推:}如果有导师、师兄师姐或朋友进行内推,是最好不过的,但需要将简历直接推荐到对应组内,若只是使用内推码则作用不大。

\textbf{校友企业:}如果因为缺乏实习经历没法直接进入大厂实习,可以考虑参加学院举办的校友招聘会(联系辅导员即可了解信息),校友企业的准入门槛不高,积攒经验后方便跳槽然后找大厂实习。

\subsection{日常实习}

再聊一下日常实习和暑期实习的取舍问题。实力足够的话,肯定是暑期实习更好。但要论性价比,还是日常实习高。尤其对于之前没有实习经历的同学,我更推荐多投日常实习。原因如下:

\textbf{门槛差异大:}当前大环境越来越卷,暑期实习难度直逼秋招,要想拿暑期实习,往往需要之前就有实习。日常实习的难度虽有提升,但还是有很多大厂是愿意包容0实习经历的同学的,尤其是非热门岗位,难度明显低很多。

\textbf{对正式招聘的影响差异小:}暑期实习和日常实习最大的区别就在于是否有机会申请转正。一方面,暑期实习未必能转正成功,如果没有转正成功或是放弃转正,在秋招时申请其他公司,和日常实习没有任何区别(也没人会在简历里专门写日常/暑期)。另一方面,家花不如野花香,很多公司对于自家转正的暑期实习会压价,因此秋招时更多人会选择别的公司,此时暑期与日常就没有什么区别了。此外,不少公司的日常实习可以申请转暑期实习,甚至可以直接申请转正,进一步减小了与暑期实习的区别。

有同学可能觉得,暑期实习会重点培养,而日常实习拿的都是边角料项目。但其实机会都是人争取来的,有想要提升自己的诉求,大可以大大方方与mentor说:“我想从这段实习里多学点东西,希望能借此转暑期/转正,希望能给我些有挑战性的工作。”大部分mentor都能理解满足大家的诉求,毕竟他们曾经也是从萌新一步步过来的。

最后,实习与正式工作有所不同的是,寻找实习时,不用过于追求work \& life balance,因为某个公司很卷就回避。一方面,实习时工作强度高,意味着单位时间内容易有更多产出,在秋招时能增加竞争力,而过于轻松的实习可能无法支撑起简历内容。另一方面,经历过相对高的工作强度后,也能通过实习来判断自己是否真正喜欢这类工作。