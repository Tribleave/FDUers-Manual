\chapter{在校准备}

这一章节主要介绍复旦大学在校期间可以利用的就业资源,以及可以为找工作做哪些在校准备。内容由不同学长姐的单篇经验分享文章构成。

如果您是想要分享相关经验的校友,并有意愿为本章节撰写经验文章,欢迎您通过邮箱:21210240339@m.fudan.edu.cn联系我,或在公众号“破蛋计划Beta”下面留言,非常感谢您的支持!

\section{本科课程成绩及相关评优}

课程成绩对于本科生比较重要,专业基础课课程成绩基本跟你在这门课上投入到有效学习时间正比,越是专注越是努力就越能取得好成绩,论文类的课程则和老师胃口挂钩,你写的好老师不一定这样觉得,但你越是合乎老师胃口越容易取得好成绩,而对于老师胃口的判断主要来源于上课认真听讲,仔细琢磨。

好成绩对于本科生影响也很多,后面几乎所有评奖评优都和绩点挂钩,很大程度也影响本科的就学体验和个人自尊。同时绩点基本决定你能否保研。但也不必盲目保研,尽可能多跟已经保研的学长学姐多交流。然后在保研时选择也可以大胆一点可以多试试更好的学校,另外提醒,尝试更好学校的前提一方面是绩点够高,一方面是及早在大一大二就参与科研有一定成果。

关于荣誉评优,每年有五月评优和十月评优,大四比较重要的是学生标兵、学生干部标兵、书院之星、毕业生之星、复旦/上海市优秀毕业生。这些评选跟一个综合能力有关,但参与这些评选绩点是默认前提,尤其是毕业生之星基本都是每个专业的第一名。