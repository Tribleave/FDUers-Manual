\newpage
\section{03届博-王真星-上海第二工业大学副教授}
联系方式:13661479214
\begin{itemize}
\setlength{\parindent}{2em} 
    \item \textbf{老师可以分享一下你的个人经历吗?}

 2003年,我从复旦大学计算机软件与理论专业毕业。那时候进入高校工作相对容易,并且当时我收到了厦门大学和华东师范大学的两个offer。在经过深思熟虑后,我最终选择了华东师范大学。

 
然而,我在华东师范大学担任教师一年后,便辞去工作,去做了两年博士后。当我博士后出站后,环境已经发生了很大变化,一些名牌大学已经不再容易进入。由于那时我自己的孩子刚出生,综合考虑后,我选择了来上海第二工业大学教书。


在我应聘的那个时期,上海第二工业大学还是一个不错的选择,学校有了一批老前辈和专家作为支撑,由于博士后的数量相对较少,因此我比较容易地进入了这所学校。但是,近两年来,虽然博士的数量不断增加,但学校的排名却不断下滑,这让我越来越感觉到学历的价值在贬值。


    \item \textbf{老师您在读博士后之前在华师大教书,为什么读完博士后不接着去华师大工作了呢?}

我在华东师范大学教书期间,我所在的学院是教育科学学院,从事信息技术方面的研究。然而,由于我的专业背景偏向理工科,与我所从事的研究并不完全对口,这也让我对这份工作感到不太满意。实际上,华师大并不十分支持我出做读博士后,他们更希望我一边教书一边在指定的合作机构做博士后。但我联系的是清华大学并且已经办好手续,因此最终选择自动离职,以追求自己的理想。


我其实并不后悔自己做的这个决定。博士后期间,我在了深圳清华大学研究院,专注于各类智能化仪表的开发领域,这是一个软硬件相结合的研究方向。在研究院,我所接触的都是与我专业领域紧密相关的内容,这让我充满了浓厚的兴趣。直到现在,我依然在这个领域深耕细作,我相信在国内高校中,我在这一领域的水平也是佼佼者。而且,我所指导的学生在步入社会后,在相同领域内所获得的薪资水平也是相当高的。


    \item \textbf{从老师经历中知道老师也有公司,想问老师现在创业容易吗?如果想创业是先有产品还是现有客户呢?}

创业并非易事。我有过开公司的经历,既要洽谈业务,又要具备足够的说服力,让谈判对手信任你,愿意投资,你需要逐一应对这些挑战。并且技术领域充满未知,我必须判断自己的技术能否解决问题,能否合理规划开发边界,并提醒客户注意潜在风险。缺乏这些能力的话,很可能难以维系公司运行。


一般来说,创业应先找到客户,再根据客户需求开发产品。这对刚毕业的学生来说却是一项巨大的挑战。学校里学到的理论知识与实际工作存在很大差距。社会上的公司所面临的问题,远比学校里教授的理论复杂。

在创业过程中,我经常遇到不熟悉的事物,因此我不断研究、学习。如今,尽管我已经50多岁,但仍然保持学习的热情,亲自参与项目。如果有研究生愿意加入,我也非常欢迎他们为我提供助力。

  
    \item \textbf{老师您是什么时候想创业的呢?创业时如何选择广告的关键词呢?}
    
实际上,我的创业经历可以追溯到很久以前,但真正意义上的创业是从2016年开始的。在那段时期,我也在积极为自己的公司做广告推广。


我们的策略主要是依据行业特点来挑选关键词,这样可以更精准地匹配潜在客户的心理和年龄层次。以我的行业为例,它侧重于设备的研发,因此选择我们的项目的大多是公司老板。由于我们的业务性质,移动端设备的使用频率较低,客户群体主要依赖电脑端进行操作。此外,我们的客户群体在性别上以男性为主,他们通常都是企业的老板。争取往这些方向来引流。

    \item \textbf{那如何做产品?如何筛选客户呢?}

筛选客户过程是需要持续升级迭代,并且伴随着严格的筛选过程。这一现象与公司的规模和所处行业密切相关,同时也与我们产品的质量紧密相连。

我们的产品必须追求精细化,因为只有达到精细化水平,才能在国际市场上与国外竞争对手抗衡。尽管我经营的是一家小公司,但我所生产的产品质量在全国范围内都是一流的。

    \item \textbf{老师您是如何平衡公司创业和学校科研考核的呢?}

我并不担心考核,实际上也没有什么问题。学校的考核方式多种多样,其中包括国家基金这一项,它与学术论文紧密相关,属于省部级的大型项目,通常与科研平台有关。此外,还有横向项目或者发明专利等考核指标。如果一个学期能够发表一篇SCI论文,就完全可以应对考评的要求。

一般高校三年一考核,包括课时、科研成果、论文、专利、专著等。如果考核不合格也是非升即走。

然而,对于刚刚进入高校的年轻博士老师来说,他们可能在起初能够发表一些文章,但随着时间的推移,可能会面临无法再发表任何文章的困境。


    \item \textbf{是什么阻碍年轻老师发不出来论文呢?}

一开始,老师们还能在学术领域受到学校的影响,发表一些文章。但随着成家立业,生活中的琐事增多,年轻教师也渐渐放下了研究工作,文章的发表也变得稀少,职称提升也遇到了瓶颈。这确实是一件让人感到非常痛苦的事情。

例如,一些刚进入高校的青年教师,在起初缺乏知名度的情况下,如果无法成功申请到项目,考核不合格,就可能面临被劝退的困境。

    \item \textbf{对于刚进高校的青年教师发论文和申请项目哪个更重要?}

最重要的是申请国家自然科学基金,这个很关键。

此外,申请各类项目也是非常重要的,其中纵向课题相较于横向课题显得更为重要。当然,横向课题的资助金额有时可能会较低。对于刚刚步入高校的青年教师来说,能够获得的项目数量毕竟有限。

当前的情况是,国家自然科学基金的审核变得更加严格,资助数量也有所减少,这使得许多申请者的处境相对较为艰难。


    \item \textbf{高校的时间是不是比较自由?相比公司企业来讲?}

在高校中,职位主要分为两大类:行政和教学。行政职位需要按时坐班,而教学职位虽然不需要坐班,但工作内容同样繁重,包括授课、辅导学生以及解答疑问等。

相比之下,公司虽然每天都需要打卡,但整体压力并不像高校那么大。当然对于自己当老板,压力还是大的。我每天一大早起床就开始工作,一直持续到晚上十一点半。我相信与我同龄的人中,很少有人能承受如此高强度的工作。

高校员工享有寒暑假,虽然空闲时间比公司员工多,但收入大约只有企业员工的四分之一。而且,多年来高校的工资水平并没有太大的变化。当然,退休待遇还是可以的。


    \item \textbf{高校老师如何平衡家庭和工作呢?}

平衡家庭与工作确实是一项挑战。特别是对于那些需要照顾孩子的女老师,她们往往很难有足够的时间和精力去进行学术研究,因此学术论文的发表和项目的申请都变得较为困难。这种情况可能会导致她们长期停留在讲师的职位上。

    \item \textbf{如果想去高校当老师,是不是读到博士后能好一些?哪些人适合进高校呢?}

如果只是希望成为一名高校教师,我个人认为没有必要从事博士后研究。博士后经历固然是一种资历,但最终学历仍然是博士学位。如果你认为博士后研究对你想去的高校职位有极大的帮助,那么可以考虑申请;否则,其意义并不大。

进入高校工作,实际上更看重的是你发表文章的影响力或者接项目的能力。而且,即便进入高校,也需要面对定期的绩效评估,如果评估不通过,可能会面临解聘的风险。

如果你在发表文章方面较为擅长,并且真正热爱科研工作,那么高校是一个非常适合你的地方。但如果你更倾向于实际应用型的研究,并且觉得撰写学术论文较为吃力,那么高校可能不太适合你。目前,许多高校的研究倾向于理论化,实际应用的研究相对较少。

我本人也评审过许多基金申请,发现真正致力于实际工作的人并不多。在高校,如果你希望脚踏实地地做事情,很可能会遭遇失败,职称晋升也可能受阻,甚至可能难以维持生计。如果这种情况长期持续,生活质量自然会受到影响。


    \item \textbf{对刚毕业的博士生来说,大厂高薪工作和稳定高校工作如何选择呢?}

实际上,我们学校非常欢迎那些在大厂工作了两三年的员工加入,因为他们具备宝贵的实际工作经验。然而,在大厂工作时间不宜过长,因为过长的工龄可能会影响他们顺利调入。

对于刚刚毕业就直接进入高校的情况,可能会遇到一些令人失望的方面。例如,起薪可能会比较低,尤其是在刚入职时,由于缺乏项目支持,撰写论文的压力也会相对较大,因此初期的生活可能会比较艰难。

尽管如此,进入高校也有其独特的优势,其中之一就是退休后的工资待遇相对较为乐观。不过需要注意的是,不同学校、不同岗位的退休待遇也会有所差异。

     \item \textbf{如果要从高校跳到工业界可行吗?公司会想招这样的人吗?}

这种做法可能不太合适。我们学院有一位负责教授硬件课程的同事,他并没有实际开发经验。当他跳槽到公司后,由于缺乏实践经验,处境变得相当被动。尽管如此,他想要重新回到高校工作却已经不容易了。因此,从工业界跳槽到高校是可行的,但反过来则不太行得通。

      \item \textbf{进高校有年龄限制吗?对是否应届有要求吗?}

目前,许多高校在招聘时对年龄有一定的限制,通常不太愿意招收35岁以上的应聘者。

对于刚刚毕业的应届生来说,进入高校工作相对较为容易。然而,对于已经毕业一段时间的往届生来说,情况就有所不同了。比如,如果你在毕业一年后想要加入高校,根据现有政策,这并不容易实现。除非你拥有博士后身份,那样或许还能获得一定的机会。但对于已经在职的工作人员而言,再次进入高校的门槛就会变得更高。


       \item \textbf{海外做博士后难吗?}

在海外从事博士后研究很难。一位美国教授曾告诉我,他们对中国博士后和访问学者的接纳态度较为谨慎,甚至有些忌讳,不会让中国学者接触到核心内容,这在一定程度上也受限于当前的国际形势,为了规避嫌疑。

此外,若在国外寻求工作机会,发展空间可能会受限,机遇相对较少,有可能长期从事单一而重复的工作。而相比之下,国内的研究机会其实颇为丰富。同时,在海外生活,还可能面临文化适应的难题。

        \item \textbf{如果目前的成果不容易进985,需要做3年海外博士后再进高校吗?}

我不建议你这样做。三年之后会更难,以后会越来越卷,建议现在投递简历。

         \item \textbf{电子信息行业未来趋势是什么呢?建议刚高考完的学生报什么专业呢?}

我建议可以首先踏入该行业,并仔细观察环境是否适合您的发展。即便是在清华大学,如果环境不适宜,也难以发挥潜力。

同时,兴趣也是至关重要的。缺乏兴趣,您将难以持之以恒。我从事的工作可能需要经历100次尝试,其中99次都会失败,需要不断跨越重重障碍。如果没有浓厚的兴趣,坚持下去将会异常艰难。

此外,能否学到真正的知识,以及项目是否能够成功实施,都是关键因素。理论与实践相结合,才能确保所学能够转化为实际成果。


     \item \textbf{如果在选择职业时,发现去大厂做生成模型虽然很赚钱,但与自己的兴趣爱好不太相关,应该如何选择?如何在经济利益与个人兴趣之间找到平衡?}

我认为经济基础必须要支撑得住你的选择。所有的学习和研究都应该与经济利益紧密结合。如果没有资金支持,你是无法开展研究的。产品开发通常依赖于项目投资,这种投资驱动着整个过程。如果没有资金作为驱动力,你就没有能力投入,也无法进入市场和形成良性循环。

      \item \textbf{对计算机和电子专业的学生而言,想从事这个行业并找到相关工作要做哪些准备呢?}

从事我们这个行业,软硬件的结合至关重要。仅仅专注于硬件发展空间有限,仅依靠软件同样缺乏长远发展潜力。因此,软硬件融合是必然选择,虽然学生们普遍反映这一领域的学习难度较大。然而,我们的行业充满希望,尽管初入职场时薪资可能不及IT行业,但随着时间的推移,发展前景十分广阔,我们拥有自己的产品,并能够实现大规模生产,这个行业并非仅限于年轻人的饭碗。

我个人对这个行业充满信心,并且坚信与人工智能的结合是未来的发展趋势。


      \item \textbf{对正在学校迷茫找工作的学弟学妹有什么寄语呢?}

探索未来如同摸着石头过河,未来充满了不确定性,无人能为你指明确切的道路。在智力水平相近的情况下,最终决定胜负的是毅力。正如马斯克一样,不断尝试,即便失败也要坚持不懈,终将迎来成功的曙光。

人生就是这样,充满尝试与挑战。回顾我的过去,也是充满波折。高考两次才得以圆梦,毕业后历经四年的职场生涯,才决定考研,又是三年后才成功。硕士毕业后,第一次考交大的博士,考上了人家不要我,我又在下半年同时考4所高校才成功。我的感悟就是,要不停地努力,抱持希望,并始终朝着目标奋进。我做研发也是大部分是曲折,但最终都可以成功。

生活也需要灵活应对。当一条路走不通时,应及时放弃,止损并尝试新的路径及时调整目标。即使遭遇挫折,只要勇敢地站起来,重新出发,机遇总会再次出现。

人生应当充满希望和梦想,如果只是为了金钱而活,那么生活将失去意义。只有追逐梦想,付出努力,人生才会变得有价值。

\end{itemize}
