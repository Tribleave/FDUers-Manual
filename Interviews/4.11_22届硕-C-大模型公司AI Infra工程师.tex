\newpage
\section{22届硕-C-大模型公司AI Infra工程师}
\begin{itemize}

\setlength{\parindent}{2em} 
    \item \textbf{可以介绍下您毕业后的工作经历吗?}

毕业后来北京一家AI视觉公司做了一年多,主要是做深度学习框架,写算子、维护框架、支持公司的视觉业务。后来公司财务状况不太好(也和大环境有关,视觉类公司普遍业务下滑),很多员工陆续离职了,业务也越来越无趣,就跳槽到一家大模型初创公司了,也是做类似的工作(框架),但是工作内容更贴近大模型。

    \item \textbf{目前薪资待遇和工作强度是怎么样的呢?}

薪资应该是高于在互联网大厂的同龄人的,可能是大模型领域初级工程师的平均水平吧,在一线城市还是能过的比较滋润的。


    \item \textbf{您当初选择这份工作,从事这个行业的理由是什么呢?}

最开始想选择一个门槛比较高的领域,所以倾向于去做 Infra 相关的工作而不是做后端或者前端这些更传统的互联网方向,因为个人认为这些方向可替代性很强并且工作强度也会很大。Infra 相对来说由于门槛更高所以可以在工作强度不是很大的情况下获得相对较高的薪水,同时天花板也会更高一些。

  
    \item \textbf{这个工作/行业有哪些最令人满意的地方?}

因为在风口,所以总是能看到很多激动人心的新技术出现,可以第一时间学习跟进,同时和行业内的大佬们交流自己也可以学习到很多东西。对于想学习最新技术、把最新的技术落地的研究人员和工程师来说这真的是一个很适合的领域。


    \item \textbf{这个工作/行业有哪些最想吐槽的地方?}

如果是在初创公司的话可能经常赶due还是要加班的,频率取决于公司业务,在初创公司节奏会更快一些,所以有时候要赶一些上线的时间节点可能连着几天甚至一两个星期工作强度都会比较大一些。另外福利这一块相比互联网大厂和外企还是要差不少的,大部分初创公司约等于没有,对于想要一个很舒适的工作环境的人来说初创公司并不合适。

    \item \textbf{这份工作/行业带给您最深感受/影响是什么?}

最大的感受是能感受到处于风口的行业的活力。可以和很多大佬并肩战斗、学习最新的paper最新的技术,这个行业的人都是充满了好奇心充满了干劲的人,这种心态能够感染我。

    \item \textbf{在你的行业中,职业晋升的通常路径是什么?有哪些职业发展的方向、机会或障碍?}

我认为程序员晋升主要靠懂技术和懂业务。两者最少得占一个,两个都占最好。懂技术指的是你熟悉行业相关的技术,了解细节、难点。懂业务是你知道公司和行业的情况以及对你的工作的影响。如果两个都懂,那你可以在核心/难点业务上取得突破自然可以走的更远。如果你觉得没有这样的业务或者机会,可以跳槽去有这样的机会的地方,是金子总会发光的。对于一些门槛比较高的方向,可能不是每个人都能接触到的,那就更应该尽早入局成为“行业元老”,这样当行业成熟时你就能享受巨大的红利。

    \item \textbf{行业近年来的主要发展趋势是什么?您预测未来几年内,(在政策/AI技术等因素的影响下)行业将会如何变化呢?}

这两年大模型行业发展真的是非常迅速。我认为接下去的几年会有更多资本进入这个行业,并且政策会更加利好。我估计之后行业会更卷,包括更快的模型迭代速度、更多的芯片支持、更多的人进入这个行业,政策上肯定也会和更多的传统企业还有高校进行合作。总的来说现在还是行业上升期,我比较鼓励学弟学妹进入这个行业。

    \item \textbf{对于计算机专业的在校生,如果将来想要从事这个行业,找到这样的工作,需要做哪些准备呢?}

我简单说一下我对这个行业的理解,对于大模型行业,大致有算法(模型)、框架(训练/推理)、硬件这几个技术上的方向,一个人不大可能同时接触两个及以上的方向(当然不是完全不可能,只是很少很少),所以有意向进入这个行业的同学可以看看自己对哪个方向更感兴趣。
想要研究模型或者算法的同学需要好好打深度学习的基础,另外还要研究下这几年的一些模型以及背后的思想,这方面我不是专家,网上有很多很多资料大家可以去研究下。
对于框架,可以理解为在硬件和算法之间的中间层,要做的就是让模型可以高效地在硬件/芯片上跑起来(包括训练和推理),想做这一块的同学可以研究 pytorch、jax、megatron-lm、deepspeed 这些框架的训练代码,推理可以看框架的一些算子的实现,主要是和大模型相关的一些算子优化。
至于硬件/芯片,主要就是设计好的编程模型让算子可以高效运行,想做这个方向的同学需要学习计算机体系结构相关的知识。
算法可以在学校实验室里学习或者自学,如果要做框架或者硬件相关的方向我建议在学校的时候去找相关的企业实习,尽早接触相关的技术。

    \item \textbf{您在找工作过程中通过哪些渠道获取相关就业信息呢?}

    脉脉、牛客、企业官网/公众号、微信/qq群、各种招聘app,大家找工作的时候一定要尽可能的收集信息,信息充足是做出合理选择的必要前提。

    \item \textbf{可以复盘下您在校招时的经历和心得吗,如果能重来一次校招您会做哪些改变呢?}

我在春招找实习的时候就比较后知后觉,开始实习比较晚(七月),秋招又犯了同样的错误(专硕赶上了写毕业论文确实也挺忙的),所以秋招的时候也比较被动,比如很多可能能拿到 offer 的面试因为比较晚了就错失机会或者莫名其妙挂了。面经就不分享了网上能找到一大堆,我只想说最重要的一点就是一定要尽早开始!不管准备好了没有,我知道绝大部分同学在找工作找实习的时候都会觉得自己没准备好所以一直拖,但这个策略其实非常糟糕,因为你永远准备不好,只有一直面试->找到自己的不足->补足短板 才是正确的思路。另外不要相信金三银四、金九银十这种说法,现在越来越卷了机会很少,所以一定要尽早开始找工作,把握住先发优势。

    \item \textbf{对那些迷茫于找工作的在校学弟妹们,您有什么寄语吗?}
    
尽早开始找工作/实习。

不要迷信老师/学校/实验室能给你多少帮助,这一点可能有点绝对(因为有的老师和实验室确实能提供很多帮助),不过据我了解在复旦大部分实验室提供不了什么帮助,这里也包括上课和做横向项目,对于找工作其实没多少帮助。如果想要想要提升自己在校招的竞争力,我建议自己课外/实验室外多花点时间学习顶尖学校的网课做做lab、多看看paper,如果条件允许的话最好可以去大厂或者知名公司实习。

通过各种渠道多看看行业和招聘动态,俗话说选择大于努力,要做出合理的选择需要掌握尽可能多的信息,但是大部分人懒得去收集信息,如果你能比别人了解更多,你就有很大的优势。

如果不确定自己想要做什么,可以都尝试一下。比如可以都学学看或者去企业实习,看一下自己对什么方向感兴趣。

\end{itemize}