\newpage
\section{23届硕-许学长-地平线后台开发}
\begin{itemize}

\setlength{\parindent}{2em} 
    \item \textbf{可以介绍下您毕业后的工作经历吗?}

我现在工作的公司是地平线(国内TOP1智能驾驶计算方案与智能驾驶芯片提供商),全称是地平线机器人,公司大概有3000人,属于中小型创业公司,属于短小精悍的类型,主要有几个方面的亮点。

其一,地平线主营业务是智能驾驶芯片,2023 年,地平线在中国高阶智驾市场的占有率为35\%,仅次于英伟达。

其二,地平线是一个软硬件结合的公司,有做芯片、算法、自动驾驶软件、AI训练等。相当于一个车上的数据怎么收集,收集之后怎么训练,训练好了之后怎么嵌入到车载芯片中的这一整个流程我们公司都有在做。

我们公司的创始人叫余凯,是第一个把深度学习引入到中国的人,是百度研究院之前的院长,2015年辞职创立地平线,开始专门做人工智能芯片,2020年开始专注于智能驾驶领域。

公司基本上没有特别强的竞争对手,像华为、Mobileye之类的公司在市场占有只有个位数,在战略上我们的公司对标英伟达。

    \item \textbf{目前薪资待遇和工作强度是怎么样的呢?}

薪资待遇来说是对标互联网大厂,拿互联网大厂举例,后端开发的白菜价是26k-28k左右,算法白菜价差不多是28k-30k左右。

工作强度方面来说,同一家公司的不同部门的工作时长和绩效压力是截然不同的。主要可以分为两类,第一类属于支撑公司基础运营的岗位,第二类属于面向客户需要产出成果的上游岗位,这两类岗位的工作节奏相差很大。

以我所在的公司举例,我所处的部门是AI基础架构部,负责整个公司机器的资源,比如有多少GPU,以及云平台、容器,模型训练任务的调度。相当于是任何一种大场景都会有的偏下层一些的技术中台。我们的工作任务不直接面对客户,是最大程度保证服务稳定性,确保算法工程师的训练任务不会挂掉,而不是想如何去创新。我们一般不会为了出业绩而加班,因为对于我们来说,不是干得越多越对,而是要少犯错。但这同时也伴随着另外一种加班形式,就是万一客户那边出了问题我们要随时去支持。我目前的工作时间是早十晚八,双休。

而像上游的一些业务部门,比如说是做自动驾驶算法的,在内部会有一个赛马机制。举个例子,比如公司要做一个比较厉害的算法,内部会分成三个团队,他们各自去训练相应的模型,最后放到市场上去检验成果。效果不是特别好的团队会直接解散,解散之后这些人大部分会被派到新的项目组中,小部分会被直接裁员。这种业务部门的人员流动性和绩效压力都会特别大。周末无休,晚上两点之类的都会很常见的。但是他们的奖金比例,base薪资都会非常诱人,只有这样他们才会有动力去卷。


    \item \textbf{芯片行业的开发和互联网大厂开发在工作内容上有什么区别呢?}

我们公司是以程序员为主,基本上没有搞设备机电之类的。相对于大厂来说,地平线不像纯互联网没有实体那么虚。相对于传统制造业来说,地平线也没有那么古板,还是以90\%的程序员为主的群体。

地平线相当于是兼容了大厂和传统制造业的优点,它既有传统互联网大厂的敏捷性、高智力密集型、扁平化的轻松氛围。比如大家都是不打卡,很晚才来公司,总喜欢熬夜工作之类的。但是我们的产出结果是更实在、更偏B端一些,我们的客户都是一些新能源车厂,比如说比亚迪、上汽、广汽大众等。这也导致了我们公司的市场知名度并不太高。
  
    \item \textbf{您当初选择这份工作,从事这个行业的理由是什么呢?在择业的时候有考虑过其他非技术类的工作吗?(比如公务员、央国企、产品经理之类的)}

首先考虑的是薪资,当时也有拿到其他大厂比如美团、B站的offer,但是薪资差距比较大,另外一个因素是考虑到地域的问题,最终选择了地平线这家公司。

我也投了银行,比如说招行信用卡,农业银行等等,大概是我在面试的时候明显感觉出来了这类企业对于技术没有太高的一个追求的状态,那并不是我自己喜欢的状态。如果是对技术成长有追求,向往自由宽松的情况下,还是尽量不要去选择国企躺平。

    \item \textbf{这个工作/行业有哪些最令人满意的地方?}
    
首先,像中小公司的技术架构,部门规模会非常小,不像大厂基础架构部就会有几千人。我们公司的基础架构只有30多个人,这意味着即使是一个校招生的身份进来,依旧可以接触到所有的核心代码。整个公司的底层原理都有机会去涉猎,这也是创业公司的一个好处。大厂的话,除非是SSP那种非常厉害的人才有机会,否则在几千人中能够涉猎到的业务内容会非常的螺丝钉。

举个例子,我们公司曾经面试过一些大厂的员工,其中有一个候选人虽然在大厂做了四五年,但是是专门解决数据库的某一个索引的问题,虽然在大厂上有立足之地,但是在市场上寻找岗位会很吃亏。

    \item \textbf{这个工作/行业有哪些最想吐槽的地方?}

创业公司的规章流程,制度性会比大厂差很多。在这种野蛮生长的阶段,如果自己不够注意的话,可能会养成一些不太好的开发习惯,运气好的话,可能会有领导来点拨,教你怎样去写比较规范。但是如果没有碰到好的领导的话,一直野蛮生长下去可能会走上歪路。

    \item \textbf{选择岗位的时候有哪些需要注意的呢?}

首先从行业上,建议大家选择比较朝阳的行业。像电商等行业已经是比较成熟的了,做起来会比较没有成就感,而像新能源自动驾驶,元宇宙这种行业还是在蓬勃发展期,每天会是有希望有奔头的,而不是无效竞争内卷抢占市场。风口行业可以让人了解的比较前沿,对自己的成长非常有帮助。

其次在选择岗位的时候,尽量选择一些能够得到扎实锻炼的岗位。不要一味的去写一些业务性的CRUD,如果在面试前打听到这个岗位比较边缘或者业务属性太强,不接触核心或者技术深度不够都可以避一下雷。刚毕业的前几年,技术深度的成长是非常重要的。

    \item \textbf{小而精的厂会不会比大厂更难出去跳槽呢?}

不论大厂还是小厂,只要干的好,都是有优势的。但是优势是不同方面的,在小厂工作四五年,因为规模比较小很容易成为某一个领域的负责人。而大厂想成为某一个领域的负责人的难度会非常大,但是大厂背景的背书是非常加分的,各有各的加分项。

    \item \textbf{在车载芯片这样的行业里,校招生的成长路径是什么?和大厂的发展路径有什么区别呢?岗位的通用性有哪些区别呢?}

发展路径和大厂来说是差不多的,但是相对于大厂来说更容易有亮点。为了招聘方便,级别都会有一个对标。我们公司的h3对应阿里的p5,h4对应阿里的p6这样。我们公司的创始人是百度的,创始员工基本上都是百度的,中间又挖了一些华为、蔚来、小鹏、理想的。相当于是和市场完全对接,人员不断流动的状态,我们公司也会有人跳槽到小米,比亚迪这样的地方。

    \item \textbf{行业近年来的主要发展趋势是什么?您预测未来几年内,(在政策/AI技术等因素的影响下)行业将会如何变化呢?这种变化对于从业者来说是一个机会还是有很大的风险呢?}

我觉得校招生选择行业,首先要观察行业的发展状态,是萌芽期、增长期还是衰落期,这个网上有很多的方法论,比较容易判断。

其次要看这个行业有没有颠覆性的技术创新,比如说元宇宙,虽然是很新型的理念,但是深挖一下会发现其实没有技术理论上的突破。而只要稍微了解一下新能源汽车,就会发现这个方向还是在稳步前进的,拥有一个比较广阔的消费市场。

再其次是从技术上去判断这个行业未来有没有潜力。我所在的公司正好出于新能源、自动驾驶、芯片三个风口叠加的产业,而新能源车的市场占有率现在是30\%,以后肯定要走向60\%~70\%,盯住增量就可以了。就像2014、2015年的时候,微信还没有大范围的普及,但是可以预测到未来会有很多人用微信。

还有就是要考虑这个业务能不能落地,落地了能不能赚钱。
  
    \item \textbf{投简历的时候有哪些需要特别注意的呢?}

永远不要放弃广撒网,因为你不知道会不会有一些小众的宝藏公司。我当时投递了50多家公司,地平线是当时最不起眼,随便投的一家公司,最后反而是最惊喜的。

    \item \textbf{大厂和小厂的面试准备有什么区别呢?}

如果是那种非常普通的小厂,就是薪资和技术水平都很低的小厂,实在找不到才考虑去。

如果是小而美的小厂,一般面试标准都是和互联网大厂对标的,按照准备大厂的方法论去准备就可以了。

    \item \textbf{您在找工作过程中通过哪些渠道获取相关就业信息呢?}

学长学姐的推荐,牛客等就够用了。

    \item \textbf{可以复盘下您在校招时的经历和心得吗,如果能重来一次校招您会做哪些改变呢?}

实习非常重要。

就业和学校是不同的,进入到业界一切就是从零开始。如果确定不想留在高校搞科研的话,越早去实习就越好。千万不要做一些自以为能加title,但是其实比较水的事情,这种加分项可有可无,如果做的比较水,哪怕是写在简历上,最终还是会被识破的。

如果实在是不能出去实习,那就自己做一些有含金量的项目,把代码都搞懂,会比去刷一个虚的title有用。

    \item \textbf{对那些迷茫于找工作的在校学弟妹们,您有什么寄语吗?}
    
只要开始尝试找工作就永远都不晚。我当时找工作也是非常拖延症,别人都春招都收到好多offer,我连复习都没复习完。结果就错过了一波又一波,然后到秋招也是一直想着开始,结果拖到将近8月中旬才开始正式的复习一些八股、准备简历等等。会一直处在这种焦虑中。在这么卷的情况下,总会有人比你早,总会有很多人抢在前面。这种焦虑是非常正常的,就像我们的论文感觉那么难写,最后其实盲审院审什么的该过还是会过的。

找工作也一样。什么时候开始永远都不晚,哪怕你整个秋招都错过了。我们当时实验室有一半的同学秋招根本就没有找到心仪的工作,结果春招找了非常好的大厂,就什么时候开始都不晚。当然不要等你拿到毕业证再开始。

\end{itemize}