\newpage
\section{23届硕-肖杨-微软上海软件开发}
联系方式:fxiao20@fudan.edu.cn
\begin{itemize}

\setlength{\parindent}{2em} 
    \item \textbf{可以介绍下您毕业后的工作经历吗?}

肖杨学长在2023年6月硕士毕业后,直接于7月开始在微软上海担任软件开发工程师,至今已有大约一年的工作经验。在此之前,他有在腾讯和微软的实习经历。

    \item \textbf{目前薪资待遇和工作强度是怎么样的呢?}

根据offerShow上的公开信息,应届生的总包薪资在40-50万元左右。在薪资方面,微软、腾讯等大公司的差距都在10万元以内。他认为,在职业生涯的起步阶段,不应过于看重5万元左右的差距,而是更注重职业发展的潜力。

关于工作强度,他认为目前的工作强度相对人性化,基本上每天工作8小时,并且有一定的灵活性,例如每周可能有一到两天可以居家办公,或者调整上班时间。此外,虽然工作要求完成任务,但不一定严格限制在8小时内,工作环境总体来说较为灵活。

    \item \textbf{您当初选择这份工作,从事这个行业的理由是什么呢?}

职业发展前景:他考虑到自己是计算机专业背景,虽然曾有过进入政府工作的机会(例如上海市专项选调的offer),但他认为在政府内部的计算机岗位,如信息科或大数据中心,在整个政府部门中更多是承担一些辅助支撑性的任务。

实习体验:他在做选择时,特别重视自己在不同公司实习的体验。他表示,腾讯和微软的实习体验都很好,而相比之下,腾讯当时正经历裁员,微软的稳定性和发展潜力更吸引他。

国际化发展机会:他提到微软提供了更多与国际团队合作和出国交流的机会,这也是他在入职后才意识到的一个重要优势。这样的国际视野对他的职业发展非常有帮助

  
    \item \textbf{这个工作/行业有哪些最令人满意的地方?}

学习和成长机会:他所在的部门有较为优秀的领导和技术团队,这使得他在工作中能够学到很多东西,特别是在技术端的成长。
灵活的工作环境:微软的工作强度比较人性化,每天的工作时间比较灵活,允许员工在家办公或者调整上班时间。同时,公司没有严格的打卡制度,只要任务完成即可,这让工作和生活之间的平衡更容易实现。

国际化平台:微软作为一家大公司,提供了丰富的资源和广阔的平台。例如,他提到作为微软员工可以有机会优先使用公司内部的先进技术资源(如OpenAI的技术),以及参加国际交流项目。他在工作中有机会前往美国西雅图进行为期半个月的交流,这对他拓展国际视野有很大帮助。

职业发展机会:他认为在微软这样的国际化大公司中,不仅有技术发展的路径,还可以选择走管理路线,这样的职业发展机会非常丰富且灵活


    \item \textbf{这个工作/行业有哪些最想吐槽的地方?}

肖杨学长在访谈中表示,和同学们相比,自己目前的工作状态还算比较理想,因此没有特别不满意的地方。不过,他提到工作后的社交环境与学校相比要更加局限,大部分的社交活动都围绕着工作展开,工作之外的社交机会较少,基本上下班后很少有其他的社交活动。这一点可能是他互联网行业工作中相对不如意的地方之一。

    \item \textbf{这份工作/行业带给您最深感受/影响是什么?}

肖杨学长提到,这份工作带给他最深的感受和影响主要在于拓宽了国际视野。特别是在美国出差的经历中,他实际接触并了解了很多关于美国和其他国家程序员的生活、技术水平,以及他们正在从事的工作。这种亲身体验帮助他更好地理解全球IT行业的动态和趋势。

此外,他提到,通过与不同国家和地区的程序员的接触,他对印度程序员在全球IT界的强大实力有了更深刻的认识,并且了解到中国程序员的薪资在全球范围内名列前茅,这也使他对程序员职业发展的全球化潜力有了更清晰的认识。

    \item \textbf{在你的行业中,职业晋升的通常路径是什么?有哪些职业发展的方向、机会或障碍?}

(1)职业晋升路径

初级到中高级职位:一开始作为普通程序员,随着经验的积累和能力的提升,可以逐步升到初级或中高级职位。通常是当你能够独立承担一个方向的工作时,就有机会晋升。

技术专家路径:如果不想走管理路线,可以选择继续深耕技术,走专家的道路。这条路径适合那些希望在特定技术领域深入发展的人员。

管理路径:当你开始领导一个方向的工作,并有初级成员跟随你工作时,如果团队或项目规模足够大,你有可能晋升为经理(Manager)。这条路径更适合那些有领导能力并且愿意承担更多管理职责的人。

(2)职业发展的机会与障碍

技术与管理的平衡:在微软这样的公司,软件开发岗位不区分前端、后端、测试或算法工程师,这意味着你需要具备全方位的能力,能够胜任多种技术任务,这为职业发展提供了更多机会。

晋升的挑战:肖杨学长提到,成为管理者通常需要至少七年的时间,而随着职业发展,晋升的难度和竞争也会增加。对于35岁仍未能晋升的员工,可能会面临一定的职业危机感,虽然在外企这种情况相对较少,但仍是需要考虑的因素。

职业灵活性:微软的工作岗位具有很高的灵活性,这不仅体现在工作内容上,还体现在职业发展的多样性上。你可以选择管理路线或技术专家路线,依据个人的兴趣和能力进行调整。

    \item \textbf{行业近年来的主要发展趋势是什么?您预测未来几年内,(在政策/AI技术等因素的影响下)行业将会如何变化呢?}

    肖杨学长在访谈中提到,近年来行业的主要发展趋势之一是人工智能(AI)技术的迅速崛起,尤其是像OpenAI的GPT等技术的推动,使得这一轮技术革命中微软等公司能够跟上时代的步伐。他特别提到,微软投资了OpenAI,并积极推动其在产品中的应用,这表明微软在AI领域的战略布局正在取得成效。
    
(1)AI技术的持续发展:学长预测,随着AI技术的进一步成熟和应用,行业内将会出现更多的AI驱动型产品和服务,尤其是在云计算、数据分析和自动化领域。这将极大地改变软件开发和IT行业的工作方式和市场需求。

(2)国际化与多元化发展:学长提到微软的CEO是印度人,并且公司在全球范围内的战略部署和技术资源整合能力很强。未来几年内,国际化和多元化将继续成为行业发展的重要趋势,特别是在全球经济格局变化的背景下,跨国公司将更加重视全球人才和市场的布局。

(3)政策和监管的影响:随着AI技术的广泛应用,政策和监管将会对行业产生重要影响。各国政府可能会出台更多的法规来规范AI的使用,确保其在伦理和安全方面的合规性。这可能会影响到技术的开发进度和市场推广。

(4)技术垄断与创新:学长还提到一些技术如NVIDIA在GPU和AI领域的垄断性优势,这种垄断可能会继续推动相关领域的技术创新,但也可能带来市场的竞争压力和垄断风险。未来几年,如何在技术创新与市场垄断之间找到平衡点将是行业面临的挑战之一。

    \item \textbf{对于计算机专业的在校生,如果将来想要从事这个行业,找到这样的工作,需要做哪些准备呢?}

(1)积累实习经验
学长强调,实习是比校招还重要的途径。通过实习,你可以逐步积累实际工作经验,了解行业需求,并让自己具备胜任正式工作的能力。尽早参与实习能使你在面试中更加自信,并能够更好地应对实际工作中的挑战。通过多段实习,逐步提升自己的能力,每一段实习都能让你更接近成为正式员工的标准。实习不仅能让你在简历上加分,也能让你在面试中更具竞争力。

(2)掌握实操能力
学长指出,像微软这样的公司在招聘时非常重视候选人的编程能力和算法题的解决能力。因此,在校期间需要特别加强这方面的训练,通过各种编程竞赛、项目实践来提升自己的编程水平。在面试中,公司往往更看重你在实习或项目中所做的实际工作内容。因此,积累真实的项目经验,并能够清晰地讲解项目细节,是获得工作机会的关键。

(3)关注求职时间节点
外企的实习招聘通常在一二月份或三四月份开放,学生需要提早关注并准备好相关申请材料,尤其是英文简历和面试准备。校招是获取正式工作的重要途径,学长建议要充分利用这一机会,但同时要有充分的准备,特别是在面试中的表现要让人感觉你已经具备实际工作的能力。

(4)英文简历与面试准备
外企通常要求提交英文简历,因此需要提前准备,并确保简历内容准确、清晰,能够突出你的优势和经验。外企可能会有英语面试环节,有时会由外国面试官主导,因此需要具备一定的英语表达能力,能够用英语流利地讲解技术问题和项目经验。

(5)充分利用学校资源
学长强调,要善于利用复旦大学的校友网络,这个平台非常强大,能在实习、求职等方面提供重要帮助。与此同时,也要在自己有能力时回馈这个网络,帮助其他学弟学妹。

    \item \textbf{您在找工作过程中通过哪些渠道获取相关就业信息呢?}

    公司官网:他提到,获取就业信息的主要渠道还是通过各公司官网。大部分公司都会在官网上发布招聘信息,包括校招和实习的职位,学生可以直接在官网上申请。
    
学生年级大群:他还提到,在年级的大群里,大家会分享各自了解到的就业信息。这个大群通常有500多人,每个人都提供一点有效的信息,整体信息量就会非常大且有用。

    \item \textbf{可以分享下您在校招时的经历和心得吗?}

    双重准备:学长提到,他的校招过程与其他同学有所不同。他在进行校招的同时,还准备了公务员选调生的考试。因此,他在9月份之前就已经完成了所有计算机领域的校招工作,之后几个月则专注于公务员考试的笔试和面试,最终在1月份决定了去微软工作。
    
实习的重要性:在校招过程中,学长发现实习经历对求职有极大的帮助。由于他有在腾讯和微软实习的经历,这使得他在校招时表现得更有信心,有的公司甚至没有让他做题,直接和他聊项目经历。这是因为他在实习期间实际参与了项目,因此对相关内容非常熟悉,能在面试中表现得游刃有余。

选择的策略:在选择offer时,学长建议要给自己留有余地。例如,如果在多个公司之间有选择,他建议不要过早拒绝其他公司offer,而是在规定的时间内做出慎重选择,以免后续因为某些不可预见的情况而陷入被动。

    \item \textbf{您对那些正在学校学习,迷茫于找工作的学弟妹有什么寄语吗?}

提前思考职业方向:学长建议研究生在研一、研二期间上课时,就要开始思考自己未来想要从事的工作。虽然不一定每天都在思考这个问题,但要时常花些时间认真考虑自己到底想要怎样的职业方向。提前明确自己的目标,对未来的求职会有很大帮助。
    
充分利用学校资源:他强调,学弟学妹们要善于利用学校提供的平台和资源,特别是校友网络。学长提到,在他的实习和工作过程中,无论是在国内还是出国交流,他都能依靠复旦大学的校友网络获得帮助和支持。因此,他鼓励大家积极利用这些资源,并在有能力时回馈学校和校友网络。

保持开放的态度:学长建议大家保持开放的心态,不要过早局限自己。他提到,自己虽然已经选择了工作,但仍然保持着开放的态度,愿意接受未来的不同可能性。他认为,在职业发展的早期阶段,保持好奇心和开放的态度非常重要。

\end{itemize}