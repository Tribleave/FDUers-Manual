\newpage
\section{23届硕-肖钒-腾讯音乐大模型算法工程师}
联系方式:fxiao20@fudan.edu.cn
\begin{itemize}

\setlength{\parindent}{2em} 
    \item \textbf{可以介绍下您毕业后的工作经历吗?}

        第一份实习是在QQ音乐的搜索组。QQ音乐是搜索和NLP都在一个中心(NLP:自然语言处理,以语言为对象,利用计算机技术来分析、理解和处理自然语言的一门学科)。

我在其中做的项目是和懒人听书联合的一个AI有声书的项目,主要做的是项目里NLP这一部分,就是通过NLP找到它这个小说里面的某句话的某个角色,从而根据这个角色的一些特点来提供语音信息。举个例子:比如说去识别这一本小说里的哪些话是哪个角色去说的,他应该有什么情感,最后让AI根据每个角色特有的声音自动朗读。

我最后在腾讯QQ音乐的搜索组拿了转正的offer并入职。

入职之后没多久调去另外一个部门做AIGC(人工智能生成内容,利用AI自动生成图片、视频、音乐、文字等),简单来说就是做一个虚拟人的应用,我主要负责聊天大模型的模块。

    \item \textbf{目前薪资待遇和工作强度是怎么样的呢?}

    薪资待遇的话按等级来说我可以算是个SSP。薪资构成是基本月薪,补贴,第一年有签字费,分两年给股票(会随着股价变化)。
年终奖一般默认是四个月,也会上下浮动,具体多少和部门有关系,实际上去年我年终奖就比四个月还高很多。而对于少的来说,两三个月都有可能。QQ音乐在腾讯属于一个中上的水平,它肯定不是最好的,微信视频号,还有王者算是最好的。我入职的时候并不是大模型的风口,如果在风口入职可能更高,这个只能看运气。

工作强度算法这边还是比较轻松的,一般十点多一点到公司,晚上没什么事的话六七点就下班了,忙的时候可能到八九点,周末双休。
业务上腾讯基本上是半年定一次OKR,做大模型的不会像搜广推那样给一个硬指标。我是在偏应用的算法组,没有发论文的需求。

之前有一段时间做一个面向海外的虚拟聊天APP叫做未伴,就是希望APP能够有更多人跟他去聊天。我这个岗位会更多的考虑怎么去应用它,而不是纯粹的做大模型,包括怎么去微调,训练,写部署逻辑等。虽然要做的事情很多,但是工期不会排的很死,基本上都是自己安排自己的时间,而开发团就会把工期排的很死,加班挺严重的,干到九点多都是常态。

    \item \textbf{计算机在校的同学选择算法岗会有两个方面的顾虑,一个是网传算法岗现在的门槛较高,竞争激烈,很难拿到offer。另一种是有的同学觉得可能算法岗相较于开发更容易被裁掉,你觉得这两种说法真实吗?}

    第一个问题卷是肯定卷的,也没那么夸张。
    
现在很多同学都太做题家思维了,一个人的水平不只是论文这一个评判维度。有论文定会肯定是一个加分项,但是不是说有论文比没有论文的就一定可以进更好的公司,获得很高的待遇。企业选择人的时候更看重的还是一个人的综合能力,比如经验、能力,在面试时候展示出的对专业知识的掌握度等。

我们组去年经历过扩张,我有负责招聘中其中一个实习生岗位的一面,我发现大家都在追求一些很奇怪的东西,反而忽略了自己的核心竞争力。我遇见过很多人确实有论文但是论文很水,这样反而会造成负面效果。举个例子,我们今天招聘大模型实习生的时候,很多都是复旦,浙大之类的名校学生,他们有的有不止一篇论文,但是论含金量很低,这会给面试官留下不好的印象。但有一些学生做过一些比较有挑战性的工作,基础知识很扎实,这会在面试中大大的加分。

对于比赛这件事情来说,在资源难以获取的一个年代,没有办法接触到一些资源的时候,只能通过打比赛去体现自己的价值。但是现在由于AI等一系列技术的发展,信息壁垒被打破,不需要靠这种单一的模式去证明自己,而是要证明在这些知识技能方面自己是专精的,有见解的。

面试整个大的趋势是轻背景,重能力。

裁员这方面来说,并不存在算法比开发更容易被裁。所有的工种都存在着裁员危机。据我观察,算法在技术食物链中还是处在比较顶端的一个位置。如果要裁算法人员,公司要考虑用工成本等多方面因素。而且在腾讯这种大厂中,算法岗不会特别的冗余,整个的结构还是比较健康,而开发和运营被裁的概率就会很大。也不是每个地方都是这样的,如果一个大模型公司80\%都是算法工程师,那算法被裁的概率肯定会水涨船高,主要是看结构是否健康。
  
    \item \textbf{您当初选择这份工作,从事这个行业的理由是什么呢?}

    读研之前想过算法很卷,考虑过转开发。

读研之后做算法挺有趣的,就不打算换方向了。
    
    \item \textbf{在你的行业中,职业晋升的通常路径是什么?有哪些职业发展的方向、机会或障碍?}

我个人比较推荐两条发展路线。

其一,通过不断的跳槽来实现利益最大化。

在一个地方干了一两年之后就跳槽到下一家公司,来获取比较高的薪资和职级。在得到比较高的职级之后,以大厂的背景去一些小厂做leader,或者去初创公司,或者选择自己创业。当然从大厂中跳到小厂做一个leader还是比较稳妥的做法。

其二,积累资本之后考公或者去国企。在达到考公限制年龄之前考公,或者去国企或者回老家。因为大家都是普通人,不可能所有人都能爬到一个比较高的位置,这个方向的人数量是比较多的。


    \item \textbf{在面对大厂随时可能出现的裁员风险时,这种压力会带来怎样的情绪和焦虑?作为在校生或者刚刚步入职场的人,您认为应该如何应对这种情况?}

互联网的35岁危机是客观存在的,接近40岁在互联网行业中就属于高龄了。我们公司有一个北航本科的同学,在公司做了很久,但是最后也是因为年龄问题就被优化掉了。互联网每天有源源不断的人进来,如果到一定的年龄没有突出的优势也没有升上去,那被优化的概率就会很高。

从一个技术工种的角度来说,要在学习上有一个很强的主观能动性,不能一直闷头干活,这样容易丧失竞争力。比如如果出了一个新的模型,新的算法,要很快的去follow,学习最新的技术,去建立属于自己的技术壁垒。

我目前并不是很怕裁员,第一个是我还比较年轻,没有很大的经济压力。第二个是我会很认真的工作,在技术上做到与时俱进,坚持学习新的理论,去看源码,跟业界的同行去进行一些交流,加深自己的技术。这样 的话可能以后就算是到了35岁危机,我也已经积累了一些人脉资源、副业这类东西。

    \item \textbf{行业近年来的主要发展趋势是什么?您预测未来几年内,(在政策/AI技术等因素的影响下)行业将会如何变化呢?}

    AI行业现在的确是一个风口,但是风口之下肯定是有泡沫存在的。去年GPT出现,紧接着GPT4发布。国内资本市场看到了AI市场,不理智的进行了大量的投资,当没有得到预期效益的时候,资本回归理智,浪潮退却。我觉得在最近几年AI发展的一个大的趋势之下,AI成为主流会是一种必然,但是中间可能会有起伏。由于防火墙的问题,国内公司也会投钱去做GPT这样的一个产品,在这个过程中可能就会诞生出大量的机会。

    而AI超级应用这些概念虽然经常被提及,但是具体什么时候可以实现其实不太明朗。在目前的这个时代,也不一定要去做算法,投身做AI也是比较好的入行时机。

    \item \textbf{对于计算机专业的在校生,如果将来想要从事这个行业,找到这样的工作,需要做哪些准备呢?}

首先要进入到比较好的学校的比较好的实验室,但是不是说在一个很好的实验室就万事大吉了,师傅领进门,修行在个人。进入一个好的实验室,接触到的东西可以帮助你打一个很好的基础,实验室背景也是找工作时候加分项,但是最终找工作看的还是自己的积累。

从过来人的经验来说,一开始进实验室最好还是要想办法做一篇扎实的科研论文,要真正参与到研发的过程中,而不是糊弄一篇一看就很水的论文。要在这个领域去学到东西,去调研,去跑实验。有一篇论文之后就有一个学术经历,是一个很好的基础。

接下来就可以出去实习。就日常实习来说,复旦的学历背景已经ok了,其实一些AI日常实习门槛并不是很高,本质上就是招人帮忙干活。这个时候如果有一个比较好的论文的话,就可以让简历比较好看。争取在实习的时候积累一些工作经验,这样一整个流程下来,不论是实习还是秋招的过程,都可以游刃有余。今年我们也遇到了很多这样的学生,背景不错,去了很不错的地方实习,基本上最后都去了一个很top的地方。

    \item \textbf{请问学长在一系列研究过程中,需要刻意的去追随一个比较热门的研究方向吗?如果一个人的研究方向比较冷门,最后找工作会比较困难吗?}

    从我的经验来看,研究方向还是很重要的,其重要性和选专业来说不相上下。
    
我们在筛选简历的时候,首先会选择CS的学生,电子自动化的之类的大多数都会被筛掉。因为CS的学生已经很多了,除非是技术大佬可以无视方向,但是大多数普通人最好是在一个方向上有扎实的基础,独到的见解。当面试方向和你的研究方向特别match的时候,面试官会特别喜欢你。

举个例子,现在也会有一些公司来挖我,挖我的方向也就是和我现在做的方向特别match的。因为对于普通人来说,短时间内转型是相当困难的。只能在一个方向上去进行积累,这也是面试官最希望看到的。如果想进一个企业,研究方向肯定要和这个企业的业务有一定的交叉,

研究方向上的match比论文的级别更重要。

    \item \textbf{您在找工作过程中通过哪些渠道获取相关就业信息呢?}

基本上都是通过官网去投递,还有一些类似牛客这样的平台,再关注一些比较大的公司的信息,这样就足够了。

    \item \textbf{可以复盘下您在校招时的经历和心得吗,如果能重来一次校招您会做哪些改变呢?}

如果重选一遍的话,我一定会打好基础之后直接出去实习到最后。

如果实验室算力不能支持大模型的训练,那最好是抓紧去实习。找工作的过程中最受欢迎的两种学生,一种是所在的实验室特别好,学生可以隔着实验室起飞;另一种是虽然实验室不是那么厉害,但是实习经历非常丰富。

找实习的时候实习的资源良莠不齐,像复旦这种学校一般老师手里都会有一些企业的资源,但是企业给学校学生做的项目往往是不是很重要的,企业不愿意花太大的精力,因此才会花比较少的钱把它外包出去。如果是企业的核心项目,是不会通过这样的一个形式去外包给学校的。但是也可能有核心项目,这个比较看运气。但是其实最好的项目组的企业项目大多数质量也很一般,所以要自己出去找质量高的实习。

最理想的状态就是研一自己能做一些东西,然后立刻去面比较好的企业,比较好的项目组的日常实习或者寒暑假实习,这样对于找工作来说的话收益是最高的。

我在毕业之前是没有中稿的,当时在实验室中做横向项目比较多,基本上实验室的每一个人手中都有横向项目,也有一些师兄手里的课题还不错,但是不是说想进去就进去的。

其实在学校里面做出来的东西也未必需要有论文的成果,能把做的东西的过程在实习招聘的过程中完整的讲出来就可以了。有一些人虽然有论文,论文可能还是a会,但是如果稍微深入的问他一些相关的概念,他就说不出来所以然了,这种情况就会让面试官比较下头。
这种一般分为两种情况,一种是他本身比较水,一种是他表达能力不强。要对写在简历里面的东西有一个深入的理解。

举个例子,之前我面别人的时候,候选人说做RAG方向,中间提到用了BGE,然后我问他为什么要用BGE,为什么必须是这样的,它的优点是什么,它的运行方式是什么?但是候选人不能够清晰的表述出来,这样就会很减分。

    \item \textbf{对那些迷茫于找工作的在校学弟妹们,您有什么寄语吗?}

    希望大家踏踏实实做点东西,别水。你学东西的时候水,到时候面试的时候面试官也会水你的。就是脚踏实地吧!

\end{itemize}