\newpage
\section{22届硕-司学长-某电池仿真公司技术销售}
\begin{itemize}

\setlength{\parindent}{2em} 
    \item \textbf{可以介绍下您毕业后的工作经历吗?}

    自2011年从材料专业毕业后,我一直投身于材料行业。我有幸加入一家世界500强的化工企业,担任管理培训生。管理培训生这一角色,大家应该都很熟悉,它通常涉及轮岗制度。轮岗有其利弊:不利之处在于,你可能无法深入接触核心业务,更多时候是在做辅助性工作,类似于实习。然而,对于职业初期的我来说,轮岗是一个宝贵的机会,它帮助我明确了自己的职业方向和规划。正是在轮岗期间,我决定成为一名仿真工程师,并从2013年到2018年一直从事这一领域。
    
2018年,我做出了一个重大决定:考取复旦大学的在职研究生学位,专业是软件工程。我这么做的原因是两方面的:一方面,我希望能够深化自己在软件领域的知识;另一方面,我渴望在职业发展上不仅仅局限于使用软件,而是更深入地参与到制造业中,真正理解客户的实际需求,而不仅仅是计算机的需求。这一转变也促使我从仿真工程师转向销售岗位。我曾认为,仅仅使用仿真软件有其局限性,而开发仿真软件则更具意义,这促使我转向销售领域。

在销售领域,我最初是在原材料行业工作,直到今年才转向仿真软件的销售。尽管原材料行业是一个基础产业,但它并不是一个快速发展的朝阳行业,而是一个相对稳定的领域。然而,我更倾向于投身于新兴行业,与这些行业的快速发展同步前进。

    \item \textbf{现在工作的公司是什么样的公司呢?}

    我目前就职于深圳的一家新能源领域的技术公司。这家公司是服务于制造业的一家软件公司,是工业用的专业软件,而不是开发通用的平台型软件。

    \item \textbf{你觉得目前的工作强度是怎样的呢?}

我最初从事的是制造业的技术工作,这个行业的工作节奏通常是稳定有序的。由于我所在的公司已经发展多年,规模庞大,它并没有实现快速增长,也没有大量的投资,因此我的工作并不是特别繁忙。然而,如果有新的投资,比如建立新厂,工作量可能会相应增加。但大多数情况下,这种投资并不多见。

相比之下,软件行业的竞争更为激烈,它不受产能限制,能够将人力资源发挥到极致。这个行业是技术密集型的,对个人的依赖度很高。例如,一个高水平的员工在996工作制下的工作产出,可能比三个员工在965工作制下的总和还要高效。

我想强调的是,工作强度是一个综合的考量因素。它不仅取决于时间节点,还要看忙碌是否有其价值和意义。在我看来,对于职业初期的人来说,忙碌一些可能是更好的选择,因为这有助于积累经验和技能。
  
    \item \textbf{想请学长介绍一下技术销售每天的工作内容是哪些呢?}

    首先,销售工作的第一步是确认客户的需求。一旦客户需求明确,我们便进入第二步:明确方案。在这一步,我们需要确定能为客户提供哪些产品或服务。紧接着是第三步:评估技术与方案的匹配度。我们需要确保所提供的方案能够满足客户的需求,如果存在不匹配,我们应该如何调整。

完成这三个步骤后,我们进行第四步:报价。基于报价,双方会进行谈判。如果谈判成功,交易达成;如果失败,之前的工作可能就白费了。日常的工作就是对这些步骤进行有效管理。

销售的本质是利用公司的资源,将客户的需求精细化,并确保这些需求能够在公司内部得到实现。销售的关键能力在于沟通的有效性。如果销售人员具备技术背景,他们可以使用技术语言与客户进行深入交流;如果不具备技术背景,他们则需要采用其他方式进行沟通。基于此,我个人更倾向于推荐有技术背景的人从事技术型销售工作。

    
    \item \textbf{做这份工作最困难和最开心的点是什么?}

我认为,在职业生涯中,最具挑战性的是如何找到一条清晰的上升通道。虽然完成公司分配的任务和做好一份工作本身可能非常容易,但在这个过程中找到上升的机会则非常困难。

如果你工作了十年甚至二十年,却发现自己仍然没有明显的职业晋升路径,你的想法和感受肯定会有所不同。如果直到退休,你都没有找到一个令人满意的职业发展通道,那时你可能会感到无奈。

最令人痛苦和困难的是,当你认为自己已经做好了工作,付出了巨大的努力,并且自认为很聪明也很勤奋,但最终却发现自己缺乏上升的空间,或者没有人认可你的努力,这无论是在物质上还是精神上都是一种打击。在这种情况下,你需要做出一些改变。有些人选择跳槽,有些人选择转换部门,或者从技术岗位转向销售岗位,甚至有人选择离开一线城市回到老家等。职业生涯是一个不断自我优化和调整的过程。

最令人感到高兴的时刻,无疑是当你发现自己的职业方向被拓宽,有了向上发展的机会。但说实话,没有人能够轻易地走得很远,职业发展总是一步一个脚印。每当你迈出向上的一步,都会感到无比的快乐和满足。


    \item \textbf{想问一下新能源岗位的薪资待遇相较于互联网大厂,相差多少呢?}

首先,新能源行业虽然本质上也属于原材料行业,但像动力电池这样的领域已经显示出其新兴行业的特征。自2020年起,这个行业在短短四年内经历了爆炸性的增长。这些公司的背景和团队大多源自传统制造业,因此它们的薪资结构和组织架构可能与传统制造业相似,与互联网行业的薪资水平并不完全一致。

然而,即使在制造业中,也存在着高薪资和普通薪资的岗位差异。以动力电池行业为例,高薪往往与那些从事电池研究的专业人士密切相关,与材料研究则更为紧密,可能与软件开发的联系不那么紧密。

在制造业中,想要获得高薪资并不是一蹴而就的,它需要与个人的职业发展相匹配。选择一个有良好发展前景且可持续的行业方向至关重要。例如,我目前所在的动力电池行业,我之所以选择它,是因为我听到公司管理层的分析,他们认为电池行业的需求将持续增长,从现在到30年后,这个行业将会发生巨大的变化,需求不会中断。

如果我们把目光放得更远,未来可能会出现核聚变、超导电池等革命性的新技术。如果我们现在的角色能够为这些新技术的发展作出贡献,那么我们的职业发展也将随之获得巨大的推动。


    \item \textbf{所以在制造业从事研究岗会有更好的未来发展?}

这个不一定。我认为研究岗位就像一把锋利的刀,使用过后可能会被搁置一旁。公司运作其实与我们撰写论文的过程相似,不同的部门承担着不同的职责。比如,有些部门负责撰写“研究综述”和“行业现状”,这可能是销售部或市场开发部门的工作,这些岗位是公司的核心,承载着公司的需求。而“原理”部分通常由研究人员来完成。“实验”部分则是实施环节,可能由工厂来执行。至于“研究和展望”,则通常由管理部门来撰写,老板来总结和规划公司今年的发展方向,以及未来五年、十年的长远规划。

因此,我们需要评估自己所在岗位在公司中的重要性。要识别哪一部分最为关键。在职业发展的过程中,每个人都应该思考如何成为最终的决策者。这是有章可循的路径。如果你只专注于需求分析、原理研究或实验操作,这些单一的工作可能无法保证你的长远发展。

实际上,每个人都是公司这把大刀中的一个组成部分,无论是研究需求的人、研究基本原理的人,还是进行实验的人。但关键在于,如何能够晋升到决策者的岗位?一旦你成为决策者,你就不再容易被替代,因为你将决定谁去谁留,而不是被公司其他部门所左右。


    \item \textbf{技术销售工作的职业发展路径通常是什么样子的?怎么看待35岁危机?}
    
销售职业的发展确实伴随着一定的风险。在大型企业中,销售人员可以晋升至高级职位,类似于古代的“将军”,甚至达到“将领”级别,但要成为拥有公司股份的“皇帝”级别显然是不可能的。相比之下,技术岗位的晋升空间可能更大,可以达到“宰相”级别。然而,这里存在一个问题:如果“将军”过于强大,可能会因为功高盖主而遭到“皇帝”的打压;而“宰相”如果因为贪污等问题也会被“砍头”。因此,在销售行业中,当你发展得非常顺利时,遭遇职业发展的阻碍是很常见的现象。

尽管销售行业存在风险,但我选择加入这个行业,是因为我认为这代表着与整个社会乃至整个行业的风险共存。个人的风险是行业风险的一部分,而行业风险又是国家风险的一部分。整个国家、行业和社会都存在风险,而销售人员往往是最早感知到这些风险并做出判断的人。例如,在动力电池行业,决策者往往是第一个知道行业风险的人,而销售人员很可能是第二个。

销售人员每天都在与客户打交道,如果行业出现问题,他们往往是第一个察觉到的,甚至在某些情况下,他们比老板更早知道。因此,销售岗位是与整个行业的风险共存的,我们可以控制这些风险并采取规避措施。这也是我选择销售职业的原因。

例如,当我判断原材料行业是一个夕阳产业时,为了规避风险,我转向了销售仿真软件的行业。这是一个规避风险和职业转向的行动。然而,作为公司内部的其他岗位人员,他们往往很难有机会接触到风险信息。

有些人可能没有意识到这些风险,仍然频繁跳槽,追求高薪,但到了35岁可能会面临不良后果。他们可能不了解行业风险,也没有人告诉他们这些风险。但行业中确实有很多大佬已经意识到风险并采取了行动,选择另一个领域去发展。

因此,每个人都需要清楚地思考,哪些事情能为自己带来长远的价值,而不是仅仅追求短期利益。我非常不鼓励仅仅为了加薪而频繁跳槽,这是一种消耗性的行为。作为销售人员,拥有技术背景可以让你在与技术部门和客户的沟通中更加顺畅,这是你的优势所在。

当然,职业规划应该如何选择,作为一个技术型销售人员,职业发展应该尽早明确。例如成为CEO。但是也不要过于勉强自己,只要能够不断接受正确的反馈就可以。当我们步入社会后,会发现一直在吃苦,可能一直不开心,那可能不是你的问题,而是因为你选择了错误的道路。正确的道路应该是你热爱的,并且你认为在未来二三十年内也能持续向上发展的。


    \item \textbf{听说销售可以积累一些渠道和资源,那这些资源对你未来有什么益处呢?}

从事技术型销售的一大好处在于,它能培养你管理和运用资源及渠道的能力。通过这份工作,你不仅能够对整个行业有深入地了解,还能掌握技术知识。公司会向你透露所有技术信息,这让你至少能够分辨哪些是关键点,哪些是次要的,以及你的技术是否可靠。尽管技术型销售是一项能够锻炼你能力的职业,但必须保持谦逊,不能期望仅凭这些渠道和资源就能实现社会阶层的飞跃。

谈到社会阶层的跃迁,我们可以借助企业运营的例子来说明:经营一家企业,不仅需要技术,还需要资金。如果只有技术而缺乏资金,就需要寻求融资。融资本质上是一群有能力但缺乏资金的人与那些有钱却缺乏能力的人合作的过程。有钱但缺乏能力的人通常属于较高社会阶层,而有能力但缺乏资金的人则属于较低社会阶层。低阶层的人通过各种手段获得能力,并与高阶层的人建立联合,获得他们的认可,共同开展事业,这可以被视为一种社会阶层的跃迁。

对于我们这些人来说,首先要培养的是综合能力,这不仅包括技术能力,还包括管理资源的能力。你获得的不是资源和渠道本身,而是管理这些资源和渠道的能力。关键在于如何从这些资源和渠道中获取价值。因此,从事销售工作并不直接等同于获得了社会阶层跃迁的通行证,而是获得了“管理资源”的能力,你才有可能利用这些能力来实现价值。

作为技术型销售人员,我们不应错误地认为这些资源和渠道是我们所拥有的,这种观念是有害的。我们并不能真正拥有渠道,我们只是领取工资的员工。但我们能够了解这些渠道的信息,并知道如何与它们打交道,这才是我们真正获得的能力。例如,如果一位上市公司的CEO对你非常看好,并希望与你合作,这种机会并不容易获得,通常需要逐步建立信任。这通常需要一个团队的支持。因此,作为一名销售人员,我需要证明自己不仅仅是销售,还能管理技术,甚至带领一个团队。只有这样,我才可能赢得CEO的信任,并有机会达成交易。


    \item \textbf{那您提到的管理资源和渠道的能力具体体现在哪个方面呢?}

这个能力,具体来说,就是能够满足他人需求的能力。那么,我们该如何做到这一点呢?首先,可以通过利用公司的资源来满足客户的需求。其次,提供有价值的信息也是一种有效的方式,这可以是市场趋势、行业动态或特定问题的解决方案。此外,协调多方资源以达成共同目标也是一种重要的能力。总的来说,最关键是在于积累信誉,也就是建立良好的名声。我们提供的方案不仅要能够满足客户的具体需求,也必须是切实可行和可以实现的。

要做到这些,我们需要深入了解客户的真实需求,然后利用我们的专业知识和公司资源来制定解决方案。同时,我们还需要保持对行业动态的敏感性,以便及时提供最新、最有价值的信息。通过这样的方式,我们不仅能够帮助客户解决问题,还能在长期内建立起客户对我们的信任和依赖。


    \item \textbf{如果想创业的话,做技术销售是不是一个比较好的切入点?}

可以说是这样的,但我们需要明确两个方向。第一个方向是加入创业公司,担任技术型销售的角色。这与自己创业非常相似,至少需要以创业者的心态来对待这份工作,全身心投入,勇于面对挑战。

第二个方向是加入大型企业。如果你选择在一家规模足够大、能够为你提供长期职业发展机会的公司从事技术销售,那么你应该考虑走管理路线。在任何一家大型公司中,管理层往往并非来自管理专业背景,而是可能来自技术、销售、采购等基层岗位,通过实际工作经验逐步晋升而来。如果你想创业,那么起点通常就是从一线工作做起,积累经验,了解业务,为将来打下坚实的基础。


    \item \textbf{那我们如何才能有更好地发展呢?}

我相信,大多数学生都倾向于技术思维,就像我们认为只要把刀磨得锋利,总有一天会派上用场。然而,现实情况可能是,未来我们可能不再需要这把刀,而是需要枪或坦克,技术可能会被迅速颠覆。特别是在软件行业,这种情况尤为明显。因此,我们应该专注于做有价值的事情,投入更多的时间是值得的;如果事情本身没有价值,那么投入再多也是徒劳。公司的发展总是伴随着有价值的和不那么有价值的事务,而那些有价值的事务往往被少数人所承担,这并不完全是因为能力差异,更多是因为机会或运气。

我们不应该只是埋头走路,更要记得仰望星空。我们必须了解全局动态,仰望星空意味着观察整个社会和公司的发展趋势,以及你周围的人在做什么。这是一个宏观视角,也是最关键的视角。当你们真正开始工作,或者真正步入职场后,就会发现,每天重复完成手头的工作其实是相对简单的。根本原因在于公司通常只把那些你能够胜任的任务交给你。只有少数人会获得成长的机会,比如公司与你共同研究,将你放在重要的位置上,这需要机遇。

如果你想要有良好的职业发展,一个更根本的问题是‘如何成为决策者’。整个公司乃至社会都可以看作是一个树状结构,这中间有许多值得深思的问题。对于没有工作经验的人来说,这肯定充满了不确定性。但实际上,没有必要过于焦虑,因为职业规划是一个终身的过程,是所有人都在持续思考的问题,不仅仅是35岁的人,甚至像特朗普这样80岁的人也在考虑职业规划。


    \item \textbf{如果计算机专业的毕业生想进入这个行业,面试中会更看重哪些能力呢?}

首先,用人单位在招聘没有社会经验的应聘者担任销售岗位时,通常会更看重应聘者的意愿。他们可能不会过分关注你的具体能力,而是更倾向于评估你是否具有与人沟通的性格特点,以及你是否展现出了对销售工作的积极态度和稳定性。

其次,进行行业调研和对应聘公司的了解是非常重要的。在面试中,你可以表达出你对公司的了解,例如:“我了解到贵公司在某某领域有着显著的成就,我认为销售岗位在贵公司扮演着至关重要的角色。”

再者,展示你与销售相关的经验和技能也很重要。即使你没有正式的销售经验,你也可以分享在校期间参与的相关活动,可以说:“虽然我没有正式的销售经验,但我在校期间尝试过一些销售工作,甚至涉足过保险销售,我对此非常热情。”

然而,我想说的是,一旦你选择了技术型销售岗位,想要再回到技术岗位可能会比较困难。相反,如果你从事技术工作后再转向销售,这个过程会相对容易。因此,我不太建议人们一开始就直接从事销售工作,而应该考虑自己的长远职业规划和兴趣所在。

    \item \textbf{现在有很多应届生即将走入社会,会有些迷茫和焦虑,你有什么想对他们说的吗?}

    我了解到复旦大学的许多同学,大约有一半的比例,选择加入互联网大厂从事技术岗位,我认为这本身是一个非常有前途的选择。同时,我也相信,从事人工智能算法岗位,从长远来看是一个可以持续发展的职业道路。
    
对于那些选择进入体制内工作的同学,我认为这也是一个不错的选择。但需要注意的是,在体制内工作,技术性的工作可能相对较少,有些可能需要从基层做起,工作中更多地涉及与人沟通和协调。

关于那些加入互联网大厂的同学,我认为互联网大厂目前可能正处于一个成熟期,不再是一个快速增长的朝阳产业,但也绝非夕阳产业,它们将长期存在。我相信这些大厂会继续发展壮大,不断在人工智能等领域进行投入,以推动业务的持续发展。但在这个过程中,对人员的需求可能不会像过去那样多。因此,我想对那些进入大厂的同学说,要么努力向核心岗位靠拢,要么提升自己的管理能力,争取进入管理层,这样可以减少被裁员的风险。

最后,我最想给出的建议是,一定要根据大环境的变化来调整自己的行为和策略。在这个世界快速变化的今天,我们需要具备灵活调整自己的能力,根据公司和团队的发展方向,及时调整自己的职业路径。同时,不要给自己设置太多限制,保持开放的心态,抓住机遇。

\end{itemize}
