\newpage
\section{24届硕-陈学长-java后端工程师}
\begin{itemize}

\setlength{\parindent}{2em} 
    \item \textbf{可以复盘下您在校招时的经历和心得吗,如果能重来一次校招您会做哪些改变呢?}

我面试的都是计算机后端岗位,公司是国企和私企为主,主要聊这个方面。

主要的遗憾:
1.没有参与中长期(3-6月)的实习,掌握实际的工作流程。(迭代,发版,线上问题排查)
2.个人没有实际对客的项目经验,导致面试很难走过技术终面。

    \item \textbf{您在校招时的收获有哪些呢?}

1.中大厂的面试要求不太一样,不同公司的要求也不同。

2.优秀的实战项目是加分项,但是没有很好的项目不代表找不到工作。

3.可以适当的润色经历,但是不能和真实工作场景的解决方案不同。

4.微服务不是那么重要,考察重点还是在你简历上描述的项目和个人学习的知识点上。你简历上提到的知识点和个人兴趣都可能被问到,要做好准备。

5.个人认为java后端考察的重点是:线上debug能力,数据库调优,消息队列底层原理,HTTP与TCP,语言的底层调优等(golang与java的复习范畴不太一样)

6.面试是长期的拉锯战,很多身边的同学7-10月都没有offer,11月才开始有收获,放宽心。当然还是需要尽早投递,过了时间就没有补录了。

7.面试时,自信的心态和流畅的沟通也很重要。对于校招生,更关注沟通能力和思考能力。

8.实习转正需谨慎,建议实习转正的时候,不要放弃面试的准备。

9.国企更看重你的项目经验,沟通能力、个人兴趣等综合素质。

10.找好搭子,及时交流信息,也有助于保持良好的面试节奏。


    \item \textbf{您在找工作过程中通过哪些渠道获取相关就业信息呢?}

国企招聘渠道:
推荐国聘(app,微博,b站),国资央企招聘平台

https://xiaoyuan.iguopin.com/

https://cujiuye.iguopin.com/job

还有一些在牛客付费卖表格的,可以用,但是表格的信息一般会晚5-10天,凑合能用。

    \item \textbf{可以推荐一些对找工作有帮助的公众号和up主吗?}

1.万诺coding(看笔试的答案)

2.小林coding,javaGuide,代码随想录等公众号及知识星球(项目可以用,但很难加分,太通用了)

3.极海Channel,鱼皮,校招vip等b站up   

\end{itemize}