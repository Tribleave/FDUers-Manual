\newpage
\section{97届本-某学长-一级市场股权投资}
\begin{itemize}

\setlength{\parindent}{2em} 
    \item \textbf{可以介绍下您毕业后的工作经历吗?}

       于复旦大学毕业之后,在金融行业工作了20多年,从事一级市场股权投资,投出了多家上市公司。
       
    \item \textbf{可以请您介绍一下PE/VC在金融行业里的工作内容吗?}

    要全面理解PE和VC在金融行业中的工作内容,需要从整个金融领域的各个层面来进行认识。目前,国内的金融行业正面临低潮,受到了政策打击和限薪的影响。
    
    金融行业大致可以分为以下几类:
    
首先是中介角色,包括四大会计师事务所、律师和评估师。主要从事财务、审计和评估工作。律师需要考取律师资格证,券商的保荐人需要了解监管法规和政策。

第二类是一级市场的投资,包括VC(风险投资)和PE(私募股权投资)。VC通常投资于天使轮和早期项目,需要找到具有前瞻性行业趋势的好团队和项目,深入了解政策和金融市场趋势。PE则专注于中后期项目,资金量大,收益率相对稳定。一级市场的项目在退出时需要上市,上市地点包括香港、国内的主板、创业板、北交所、科创板以及美国市场。上市过程通常需要公司寻找中介机构的帮助,并将股份卖给公开市场的投资者。

二级市场主要涉及股票买卖,跟踪公司和行业趋势,选择好的公司进行投资,并为公司定价。这对理工科背景的学生很适合,因为需要分析行业趋势并进行资产定价,这是金融领域的核心技能。

此外,还有量化投资,通过模型跟踪市场数据,设定交易模型,让机器自动交易。这与计算机和数学紧密联系,适合用数学方法进行投资、市场波动分析并赚取收益。

因此,PE和VC的核心工作内容包括找到好的投资项目,进行项目定价,管理和退出项目。工作强度很大,需要不断学习和判断市场变化,对行业趋势和政策的把握非常重要。


    \item \textbf{请问学长,您做PE和VC这份工作最让您感到满意的点和最想吐槽的点分别是什么?}

    在PE和VC的工作中,最让人感到满意的是获得感。作为甲方,你始终在主导位置,别人来找你融资,这让你能够接触到各行各业,了解不同领域的动态。
    
而且投资工作真的很有趣。相比咨询行业的工作,投资更让人有成就感。咨询工作是为客户服务,很多企业知道自己的问题,但自己解决不了,就需要找外部的咨询来协助解决问题。而PE和VC则不同,是始终都在主动跟踪行业,投进去后要密切保持联系,指导团队成长。随着经验和资源的积累,会有很高的成就感。同时金融行业的主要目的是优化配置社会资源,通过市场化的方式进行资源的优化配置,这也很有意义。

从个人角度来看,这个行业让人很容易获得价值感。每天的工作都不一样,总有新的东西可以学,有新的价值可以呈现给社会。这是这个行业最好的地方。

然而,这个工作也有让人感到无力的地方。比如政策风险,前两年教培行业被关停,有投资人当时坚决看好教培行业,不愿意退出,结果政策一变,整个投资项目失败了。这种不确定性让人感到无力。还有宏观经济环境的变化,如中美关系紧张等不可控的因素,或者人事变动等,都可能让本来不错的项目变得糟糕。二级市场的交易也是如此,每天都要面对边际变化,但有些事情你无法控制。

这是人生的一部分,必须学会适应和接受,然后在自己能努力的范围内去做。如果因为自己不够努力,研究不够,判断错误而失败,可以通过提高认知来弥补这些不足。先尽人事,然后听天命。

总的来说,PE和VC的工作既有获得感和成就感,也有许多无法控制的不确定性。这是这份工作的两面性。
  
    \item \textbf{近些年来和未来几年内,政策和技术等因素对金融行业的主要发展趋势是什么?这个行业将来会如何变化?}

    近期,金融行业受到政策的重大影响。对整个金融机构的薪资进行了限制。短期内,这些政策肯定会对金融行业产生很大的影响。然而,有些基本规律是亘古不变的。金融行业之所以被认为是好的行业,是因为它总是与资源和财富紧密相连。就像屠夫的手上总是有油,因为他天天在屠宰。金融行业也是如此,核心从业人员的收入永远是高的。只要你掌握核心资源并能产生核心影响,你就会得到与之相匹配的收入。
    
虽然短期内肯定会限薪,但不需要过于担心收入问题,关键是看你能为行业和公司创造多少价值,或者你离核心业务有多近。

关于AI技术的进步,的确会导致低端岗位被淘汰。比如,低端画图的画师和低端的码农将会被淘汰。然而,好的算法人才仍然稀缺,包括做量化策略的人也是如此。AI永远只是一个工具,它会淘汰一些低端的从业人员,减少一些工作机会,但也会打开新的领域,创造新的机会。因此,不需要担心技术的进步,重要的是不断适应和学习变化,成为最能创造价值的人。

总的来说,虽然短期政策会对金融行业有影响,但长期来看,尤其是贴近业务的岗位,仍然非常有前景。同时AI技术也会带来新的岗位和机会,不会对从业者造成太大的打击。对于个人来说,不断适应和学习,成为能够创造价值的人,是应对未来变化的关键。

    
    \item \textbf{对于从事这一行业的新人来说,他的职业晋升路径是什么?有哪些职业发展的方向和机会或者障碍?}

新入职的新人通常会从分析师或VP(副总裁)开始。最初的工作可能包括行业研究、投资报告、行业分析报告和会议纪要等。在VP阶段,可能会协助ED(执行董事)或MD(董事总经理)做项目,跟随他们学习经验。之后,逐渐独立负责项目,从ED升到AMD(助理董事总经理),这个过程中需要积累资源和判断能力,paperwork的事情可以交给下属去做。

最关键的是要热爱投资行业,并且不断学习,吸收新的知识和行业变化。聪明和勤奋也是必不可少的。行业中有很多聪明又勤奋的年轻人,他们成功得很快。如果只想着混日子、赚点中介费或回扣,是没有前途的。

你必须真心喜欢你所做的事情,这样才不会觉得辛苦,才能长期坚持做下去。如果你热爱这个工作,哪怕加班加点干活,也不会痛苦,反而会很有成就感,就像打游戏一样。

如果你觉得上班很累,只想着摸鱼早点下班,那你不适合做这份工作。所以一定要找到让自己感到有成就感的事情,这很重要。只有真正喜欢这个行业,才能在不断的挑战中获得正反馈,取得成功。


    \item \textbf{请问学长,对于计算机专业的同学,他们如何进入金融行业从事PE或者VC这样的工作呢?需要做哪些在校的准备?是否建议他们再去读一个MBA这样的金融学位?}

    我认为最好的组合是本科读理工科或者计算机,然后硕士读金融学、经济学或者MBA,再进入金融领域。现在很多PE和VC公司为了发展硬科技,喜欢招一些相关领域的博士,比如医药学、化学等专业的人才。但我个人不建议太把自己限制在某个专业上,因为投资的风口是不断变化的。
    
过去几年,我们经历了很多商业模式的创新,比如阿里、头条、腾讯等都是商业模式创新。之前是一些基础产业的发展,比如钢铁、汽车等。所以,每一波机会都在变化,如果你只专注于某个行业,过几年可能这个行业的机会就没了。现在很多搞医药的人就很糟糕,因为从宏观上看医药这个行业不行。

关于在校准备,不同的路径需要不同的准备。如果想要做的是偏概念的分析,更多的工作内容是对行业的宏观分析;而量化分析则是看数据、财务报表和增长。如果你是想做量化,那整天需要考虑的是数据回撤、模型和策略,纯粹跟数据打交道。而PE/VC则更多是跟人和概念打交道。

如果一个同学有志于从事金融行业,首先要知道自己的兴趣在哪个方向,了解这个行业里的不同岗位的具体工作内容。要有专业和禀赋的准备。以前进入PE行业时没什么背景要求,但现在通常需要有做LP(有限合伙人)的条件,要么很有钱,要么有资源,进入很难。
目前,计算机专业的学生进入金融行业,主要是去银行和投行做金融科技岗。但这些岗位未来发展有限,因为本质上还是码农,做IT系统的工作。转到业务岗才有发展。

如果你是计算机背景,最好能转到业务岗,比如做量化策略,这样有更好的发展前景。金融领域的核心价值是投资和募资,尽量把自己往核心业务方向发展,这样才有更好的职业前景。

    \item \textbf{对于计算机专业的同学,进入投资行业有一定壁垒。平时能接触到的金融行业校招渠道主要是银行、国企、金融监管等单位的金融科技岗,离金融业务比较远。在这些单位和岗位中工作,未来是否有机会做投资?这样的路径设计是否合理?学长有什么建议吗?}

    首先,监管机构非常难进,但一旦进入,前途远大,如果家庭的经济状况比较富足,这是非常好的选择。商业银行则稍微差一些,因为未来可能会面临很多坏账问题,对其前景我持保留态度。很多人进入商业银行后,可能一直做柜面工作,所以对商业银行的前景我持谨慎态度。
    
如果想要转到业务岗,这也是有可能的,但机会相对有限,需要一些机遇。如果你从银行转到其他金融机构的业务岗位,可以考虑量化部门,因为银行也有交易部门,这属于商业银行的投行部门,不是纯粹的商业银行概念。

如果你有商业银行的工作背景,想要转到企业,可以考虑去财务部或做财务总监,更多的是利用你对银行政策的了解来获取贷款,这也是一种核心技能。但这就属于另外一条职业路线,不是投资线了,不对企业经营负责。

    \item \textbf{对那些将要去找工作的在校学弟妹们,您有什么寄语吗?}

临渊羡鱼不如退而结网,先多花时间了解自己,提前想清楚自己要过一个怎样的人生,是更重要的事情。

\end{itemize}
