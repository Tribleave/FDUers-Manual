\chapter{访谈录}
本章节通过采访问答的形式整理了前辈们的工作经验,内容涵盖他们的从业感受、对行业的见解,以及对当前正在求职的同学们的建议。目前,我们共整理了19篇访谈录,其中大部分是通过视频采访记录整理成文,少部分是由受访者直接填写访谈表。

目前,绝大多数受访者都是\textbf{复旦大学计算机专业毕业的校友},每篇成稿都根据受访者的要求进行了一些\textbf{匿名化的处理},且最终\textbf{都经过了受访者的审阅与确认}。希望能为读者在择业和选岗方面提供一些感性的经验参考。


我们想联系到更多元的受访者,特别希望能采访到更多各行各业的\textbf{资深从业者}!以及\textbf{相关行业的面试官},或是有\textbf{工作经验的计算机专业校友}。
如果这篇册子有幸与您相遇,并且您愿意接受采访的话,欢迎通过邮箱 21210240339@m.fudan.edu.cn 或微信公众号“破蛋手册Beta”与我们联系。非常感谢您的支持!


\section{08届本-段新杰-谷歌售后服务经理}
联系方式:duanxinjie119@hotmail.com
\begin{itemize}
\setlength{\parindent}{2em} 
    \item \textbf{可以介绍下您毕业后的工作经历吗?}
   
    A: 我是复旦本科毕业,之后去的芬兰阿尔托大学读的硕士。
在芬兰读硕士期间的最后一年会在当地选择一个企业,企业会提供给你一份工作,就叫master thesis worker。然后在企业中一边工作一边写论文。这份论文的内容对企业也会有一定的帮助,最后论文会同时提交给学校和企业。

1.我去的是芬兰当时最大的企业诺基亚的IT部门,经过了大概是半年的时间来完成我的毕业论文,之后继续在诺基亚工作了一段时间,但很不巧的是那个时候微软把诺基亚收购了,因此后续相当于我也在微软工作了几个月。然而,微软收购了之后就进行了一些裁员,当时我所在的整个team就都被裁掉了,至此就离开了微软(诺基亚)。

2.接下来我就去了华为在芬兰的分公司做售后服务经理,工作了三年多,主要工作内容是管理那些直接去给消费者提供售后服务的服务供应商。比如说客户的手机出问题了,那需要去找到华为的服务门店去做维修。这种给消费者提供售后服务支持的具体的工作是由服务供应商来提供的,那我的角色就是管理这些供应商。我在华为工作了大概有三年,我会把在华为的工作定义为我的第一份工作,之前在诺基亚(微软)其实都算是实习或者是做毕业论文,等于是part time一样。华为的企业文化让我成长得非常快,同时心理压力很大,在华为的三年后,我觉得可以去挑战一下不同的职位,于是我主动选择了去一个创业公司。

3.这是一个在芬兰做硬件产品的创业公司,工作了两年多,由于是在创业公司,因此主要工作内容基本覆盖了研发,售前,售后,客户对接等等非常广的内容。最后由于创业公司风险比较大,公司的现金流之类的问题比较多,加上公司的经营情况不是很理想,于是在19年的时候我就重新开始找工作。

4.很幸运地找到了谷歌在伦敦的售后服务经理的岗位,工作了三年多,在2023年我主动申请调回了上海,现在是在上海的谷歌公司工作。

    \item \textbf{目前薪资待遇和工作强度是怎么样的呢?}
    
(1)在同样的一个公司售后服务经理和软件开发岗位比起来薪资会低30\%到40\%。
    
(2)工作强度方面,在华为工作时,由于工作地点在芬兰,因此双休是有保障的,朝九晚六,偶尔会加班,但这只是芬兰的情况,因为芬兰跟国内有5—6个小时的时差,我已经下班了,但是国内同事可能还没有下班。针对华为的话,国内跟海外的情况差距还是蛮大的。
在谷歌的话,work life balance会好很多。基本上就是朝九晚五,周末完全可以不用看邮件,我们的manager特别强调周末不要看邮件。员工彻底地跟工作分开,放假的时候就好好放假。

当然也可能是谷歌在这方面做得比较好,其他的一些外企比如说亚马逊可能也会存在加班这种情况。工作强度主要跟公司文化相关。

    \item \textbf{您当初选择这份工作,从事这个行业的理由是什么呢?} 
    
    A:我不是那么喜欢只跟代码打交道的工作,我更喜欢跟人打交道,所以对我来说,我会比较喜欢这样一份更多地和人一起打交道的工作。
    
    \item \textbf{这个工作/行业有哪些最令人满意的地方?}
    
    A:最满意的是在华为做售后客户服务经理的时候,华为内部有很完善的体系与制度,这让我在短时间内成长得非常快,只要按照他的流程去执行,便不会犯太大的错误,可以让我一个刚毕业没多久的学生直接去负责一个国家的售后业务。

我以前对硬件产品的生产,研发,制造,销售等完全没有概念,但是在华为短短三个月我就全部都学会了。虽然那段时间压力很大,但在华为工作的时候对我的成长很有帮助,能够让我很快地掌握一个领域的知识。

在谷歌的时候,最让我感到开心的是谷歌它算是在硬件产品相对比较新的一个玩家,很多体系是不完善的,这使我有比较大的空间去按照我自己的想法去建立一些流程,并且完善一些项目。


    \item \textbf{这个工作/行业有哪些最想吐槽的地方?}

    A:第一点就是售后服务经理对整个供应链的节点都需要有一定的了解,接触到的事情就会非常的杂。每天会有各种各样不同的小事情发生,涉及的面很广,发生的问题的点很多,所以需要你有很强的同时处理事情的能力。

第二点是在售后服务经理的工作中,虽然不直接面对用户,但通过与供应商的交流,会了解到用户使用产品的实际情况。你可能会发现用户并不按照产品设计者的预期来使用产品,而是根据自己的需求。例如,产品设计是移动WiFi,但用户可能只把它当作家用路由器。这种情况可能导致产品设计与实际使用场景不匹配,需要处理各种意外情况,并保持包容心态。另外,即使进入了看似组织良好的公司,比如谷歌,也可能面对部门杂乱无章、缺乏体系的情况,需要像在创业公司一样逐步完善。这与之前的预期可能有一定差距。


    \item \textbf{在售后服务经理的行业中,职业晋升的通常路径是什么?有哪些职业发展的方向、机会或障碍?}

    售后服务经理属于运营的框架下。

发展路径分为两个方面,一个方面是垂直的,比如说我一开始是负责芬兰的售后服务,那我可能继续扩展我的业务,比如我可能负责整个北欧,继续扩展整个欧洲,再到全球。

另一方面是到相关的部门里面。比如说我现在是在做管理售后的,那我也可以去产品质量部门发展,因为会有一些共同的需要用到的东西。比如说:客户反馈的一些产品质量问题是通过售后服务经理进行收集,并反馈给质量体系的。所以我们也可以很自然地去做质量检查的事情。

售后服务经理对于整个业务流程都比较熟悉,了解得比较广泛,如果对供应链上哪一个点感兴趣,就可以跳到哪个方向。

    \item \textbf{在面对大厂随时可能出现的裁员风险时,这种压力会带来怎样的情绪和焦虑?作为在校生或者刚刚步入职场的人,您认为应该如何应对这种情况?}
    
你已经是复旦的毕业生了,已经超过同龄人很大一截了,对于大部分的同龄人来说,你已经是个人中翘楚了。你之后的人生肯定会遇到起起伏伏的状况,一定要相信自己,你是从复旦毕业出来的,那就证明你是有足够的能力的。之后可能会有不同的困难,但是一定要相信这些困难是暂时的,最后都会有不错的结果,这是我自己从身边同学朋友的观察出来!对于这些本身条件素质就很好的人来说,哪怕会经历短暂的一些困难,但最终都会顺利地走过去的!

至于说短期的,不管是裁员还是失业的风险,我觉得就是顺其自然,找工作的时候,一直想着会不会裁员没有太大的意义,就正常去做。

哪天被裁了或者公司倒闭了或者其他的情况,没有关系,继续休整一下,再重新出发,重新再找工作就好了。大厂的话,离职的赔偿还是不错的,最起码能保证你一段时间不会因为生计而担忧。

其实有时候裁员还可能是好事情,比如说我有一个目前在谷歌的同事。他之前在几个外企做硬件,先后可能被裁了三次,但每次都是就一两个月就找到了工作,拿了离职赔偿之后,可能还赚了很多。

    \item \textbf{行业近年来的主要发展趋势是什么?您预测未来几年内,(在政策/AI技术等因素的影响下)行业将会如何变化呢?}

    近年来,行业的主要发展趋势之一是越来越多的大型互联网公司开始关注硬件领域,希望通过拥有自己的硬件产品接口来掌控用户体验。例如,像苹果这样的公司成功地将软件和硬件整合在一起,从而实现了对用户体验的全面掌控。另外,像谷歌这样的公司,尽管推出了安卓系统,但由于其开放性,导致各个手机厂商对系统进行了各种定制,但是升级的情况也不尽如人意,用户可能还用着几年前的版本,因此谷歌也开始涉足硬件领域,以便更好地控制用户体验。

此外,随着人工智能技术的不断发展,越来越多的公司开始意识到依赖第三方的硬件解决方案并不是最佳选择。因此,他们开始考虑自主研发芯片以及提供自己的算力,以满足公司内部的需求。软硬件结合成为互联网大厂的一个重要发展方向。

智能硬件设备行业并非一直处于高速发展的状态,而是逐渐趋于稳定。类似于个人电脑和智能手机行业一样,经过一段快速发展期后,智能硬件设备也逐渐进入了一个相对稳定的阶段,包括出货量和产品迭代速度都在逐渐稳定。因此,整个智能硬件行业也在朝着更加稳定的方向发展。

未来几年内,我预测行业将会继续注重软硬件结合,大型互联网公司将继续探索自主研发硬件的可能性。可能某个产品处在高速发展,但整个行业将会处于一个相对稳定的状态。

    \item \textbf{对于计算机专业的在校生,如果将来想要从事这个行业,找到这样的工作,需要做哪些准备呢?}

    我认为,表达能力非常重要。能够清晰地传达自己的想法和经历,让面试官迅速理解你的意思,是至关重要的。有时候,面试者可能问你一个问题,你可能想要表达很多内容,但却不能够清晰地传达出来,这会给面试官留下不好的印象。实际上,你可能做了很多工作,但如果你无法有效地表达出来,那在面试中就会显得不足。相反,即使你的工作内容没有做到100\%,但如果你能够完整地表达出来,那么你就会给面试官留下更好的印象。
    
    \item \textbf{您在找工作过程中通过哪些渠道获取相关就业信息呢?}

    1.朋友介绍:第一份工作是通过朋友介绍得到的机会,朋友得知我在找工作后联系到了招聘方。

2.招聘网站和招聘公司:在想要跳槽或者找新机会时,我会浏览一些招聘网站上的求职信息,或者直接关注一些招聘公司的招聘信息。这是比较传统的渠道,但也是我获取信息的重要来源。

3.公司官网:对于想要加入的大厂,我会直接去他们的官方网站查看招聘信息。这是我认为最直接、最快捷的方法。

4.内推:如果有朋友能够内推,那就更好了。因为据我了解,内推可以确保简历能够被真人的招聘人员看到,而不至于被简历筛选机器刷掉。

    \item \textbf{可以复盘下您在校招时的经历和心得吗,如果能重来一次校招您会做哪些改变呢?}
    
    选择合适的路径非常关键。我觉得我当时选择的路径还算是比较合理的,先到一个像华为这样的大型企业去提升自己。华为拥有相对完善的管理体系,而且在业务领域也有一定的知名度。当时正值智能手机市场蓬勃发展的阶段,华为处于上升期,这给了我很好的发展空间。

持续学习和成长至关重要。在华为的那段时间,我认为学习新东西、不断提升自己是最重要的。这段经历为我奠定了坚实的技术和管理基础。

一旦积累了足够的知识和经验,可以考虑更适合自己的发展路径,比如创业公司或者外企。这样的选择可能会带来更好的工作生活平衡和更大的发挥空间。

对于大多数学生来说,刚毕业时的想法可能并不是很清晰。初入职场的年轻人先选择成熟的有前景、处于上升期的企业,快速提升自己的能力和经验。等到有了一定的积累之后再考虑其他选择,这样更为稳妥,风险也相对较小。

    \item \textbf{对于刚毕业的同学想进入这个岗位,是先去技术部门做几年技术工作,还是入行之后直接参与到售后服务经理这样的岗位更好呢?}

    在选择职业方向时,更重要的是考虑行业的选择,如新能源、医疗、金融等大型行业板块。

具体行业内的专业分工相对来说并不是十分重要,因为跨越不同行业的转换通常是相当困难的,尤其是在你在某个行业工作了数年甚至十年之后。这样的转换门槛极高,我身边很少见到成功从一个行业完全转移到另一个毫不相关的行业的例子。

对于毕业生的行业选择来说,选择一个未来具有长期发展潜力的行业至关重要。至于在行业内从事什么职务,如研发、售后、销售等,我认为并不是那么重要,主要看个人是不是特别感兴趣,因为以后还会有调整的空间。

    \item \textbf{对那些迷茫于找工作的在校学弟妹们,您有什么寄语吗?}

    我觉得就用我之前说的吧。都已经是复旦毕业的学生了,一定要对自己有信心,所有遇到的困难其实都是暂时的。长期地来看,你已经是一个很优秀的人,未来一定发展不会差的。
    
\end{itemize}

