\newpage
\section{23届硕-佩奇-互联网大厂产品经理}
\begin{itemize}

\setlength{\parindent}{2em} 
    \item \textbf{可以介绍下您毕业后的工作经历和实习经历吗?}

    A: 我是末流985计算机类的本科,硕士是在香港读的。关于实习经历,我其实并没有很多,但我有一段很深刻的竞赛经历,这段经历对我的职业发展产生了重要影响:
    
在本科大二大三期间,我积极参加了学校举办一些软件开发和创新创业类的比赛。那时,我注意到了校园内的一个痛点问题:学生普遍不知道校巴的具体位置和发车时间。基于这个问题,我组织了一支跨学科团队,包括计算机、设计和商科的同学,共同开发了一个解决方案。我们的项目不仅赢得了全国性的奖项,而且至今仍在线上运行。这段经历为我后来担任产品经理的工作打下了基础。

本科毕业后,我前往香港深造。在留学期间,我并没有立即准备找工作,直到接近毕业的那一年,我才意识到需要找工作并开始投递简历。至今,我已经在四家公司工作过。

首先,我在一家小型创业公司开始了我的职业生涯。然而,在我即将入职时,我接到了一家互联网大厂的面试邀请。经过2轮业务面+1轮交叉面和hr面试后,大约一个月左右,我收到了这家大厂的offer。由于那时我已经在创业公司实习了一个月,因此我决定辞职并加入第二家公司。

第二家公司是一家知名的互联网大厂,做面向开发者的平台型产品经理。这段实习体验了小团队作战的工作方式,与后面的大团队拧螺丝的工作方式有所不同。临近9月,我开始正式准备秋季招聘,赶在提前批的尾声投递简历。

在准备秋招的过程中,我投递了超过100个岗位,参加了30多场笔试和20多场面试。期间,我收到了包括百度、拼多多、vivo等在内的多家大厂的offer。在收到这些大厂的offer之前,一家专注于骨传导耳机的公司给了我一个offer,并希望我尽快入职。由于当时我没有其他offer,于是决定先加入这家公司工作。然而,在我入职两周后,拼多多和百度等互联网大厂都与我联系商讨薪资问题。最终,我选择了现在这家公司,担任互联网金融科技类产品的产品经理。

面试我现在工作的这家公司时也发生了一件有趣的事情。当时我正准备参加本科的一场公司宣讲会,路上突然接到这家公司面试官的电话要求进行面试。我匆忙找到一个窗台,打开电脑开始了面试。第一次面试只进行了不到半小时,结束后我询问了岗位详情,但hr并没有给出明确答复。大约一周后,我进行了第二次面试,面试官是一位资深老板。面试结束后,我再次询问岗位,老板也没有直接回答,而是询问我想从事B端还是C端的工作,我选择了B端。随后,我进行了第三次面试,大约在十月中旬结束了整个面试流程,并最终获得了offer,一直工作至今。

    \item \textbf{刚你提到面了三次面试,那这三次面试都重复这样的流程吗?}

    A: 通常情况下,我们会重复这样的面试流程。然而,随着面试次数的增加,我注意到低职级的面试官往往提问较为标准化,互动较少,他们更注重考察你的逻辑表达能力。而高级面试官则倾向于提出更加开放性的问题,并且态度更为亲切随和,使得面试过程更像是一场轻松的对话。可以说,面试的前半段像是严肃的考试,而后半段则给人一种轻松的感觉。

    \item \textbf{计算机专业出身大多选择做技术开发和算法,那你做产品经理的动机是什么呢?}

    (1)根据我个人性格,与其安静的写代码,我更喜欢跟人打交道;
(2)根据相关经历,我本科时就有带同学做项目的经历,并且认为很有趣。
(3)我认为程序员不能做一辈子,可能到35岁就要转岗去干管理或者产品经理,那不如尽早的接触这个行业;
(4)我未来挺想创业的,且产品经理可以积累带团队的经验,因此我也想体验一下;

  
    \item \textbf{你收到了很多家offer,后来为何选择你入职的这家呢?}

    我认为有四点原因:
(1)地点在杭州,位置很好;
(2)公司平台很大,也很好;
(3)这家公司业务也不错,属于核心业务;
(4)薪资待遇也符合预期;
    
    \item \textbf{互联网行业是不是都不允许透露薪资数据呢?产品相对开发薪资水平能差多少呢?}

对的,如果大家想了解薪资的话可以去offer show公众号上去看,相对比较准确。

如果在同一家公司,我粗略的估计,产品岗的薪资是开发岗的80\%。

    \item \textbf{你工作的这几家公司的公司作息和强度是怎样的呢?}

    A: 第一家是一家创业公司,每天早上9:30开始工作,晚上6:30结束,员工需要打卡签到,但工作氛围并不紧张,也不是很卷。
第二家公司是知名的互联网大厂,上班时间是早上10:00,中午有2小时的午休时间,下午2点继续工作,直到晚上8:00下班。周三周五可以6点下班。具体的工作时间也要看老板的工作时长,若老板加班,通常员工也会加班。

第三家公司是做耳机的公司,员工需在早上8:45前到达,晚上6:15即可下班,中午休息一个半小时。公司文化很好,鼓励员工保持健
康,所以我们常常在晚上相约跑步。近期听说原部门工作强度上来了,也没办法准时下班了。

目前所在的大厂公司从事金融科技业务,规定上班时间是早上9:30,晚上6:30下班,中午休息一个半小时。不过,不同团队的工作状况差异较大,我们组的业务需求较多,因此我经常加班,有时甚至要工作到晚上十点、十一点钟。不过,周三和周五可以稍微早点下班。

总的来说,小公司通常需要打卡,而大公司则相对自由,打卡制度较为宽松,稍晚一些到达也是可以接受的。

    \item \textbf{那你认为这份工作比当初想象中的快乐吗?工作中哪些是你比较满意的地方和吐槽的地方?}

满意之处主要体现在两个方面:首先,对于目前的薪酬水平,我感到相对满意;其次,大厂的工作环境也比较优越。

然而,需要吐槽的地方很多:首先,工作繁杂,几乎没有时间进行独立思考,每天都被各种琐事所困扰;其次,我感觉所从事的工作较为枯燥,每天只是被动地承接各种需求,缺乏创新元素。由于我们平台的产品经理主要关注的是降本增效方面,只有深入理解平台的不足之处,才能有效提升效率。但鉴于我只工作了半年,对整个系统尚未有深入的了解,因此难以实现效率的提升,这让我感到非常痛苦。第三点,我对公司业务的理解不够深入,尤其是在金融资金链路这一块,它非常专业,我常常感到困惑。最后,融入团队并进入正常的工作状态对我来说也是一件比较痛苦的事情。


    \item \textbf{产品的杂事体现哪些方面呢?}

早上一上班就会有很多人来找你,还会有一些一些令人头疼的纠缠。然而,我还没有熟练掌握应对此类纠缠的技巧。每当有人向我提出需求,我都乐于助人。但与我共事的资深产品同事们,他们擅长于这种周旋,往往一番争论后,需求便不了了之。

产品经理常常容易成为替罪羊。由于需要与多个部门进行协调,一旦进度出现不同步,责任往往会被推到产品经理的身上。

    \item \textbf{那这几家公司的工作状态是怎样的?会有摸鱼等自由的时间吗?}

在第一家公司的实习期间,由于公司的工作节奏较为缓慢,我的日常职责主要是进行数据录入和表格整理,以及处理一些较为琐碎的事务,这使得工作显得有些单调乏味; 

而在第二家公司,工作节奏适中,主要面对的是自发性的需求,没有明确的完成时间限制; 

到了第三家公司,工作环境依旧轻松,主要负责为其他部门提供业务支持,例如开发仓储系统等。在这里,我学到的东西相较于第一家公司有所增加;

第四家公司是目前所在的公司,需求往往非常紧急,每两周就有一个更新迭代,同时也有许多业务部门提出需求,需要我们迅速完成相关工作,因此我的工作变得相当忙碌。

    \item \textbf{工作中或者身边有没有产品经理的同事面对被裁员的压力呢?}

    我的一位同学在一家知名新能源车企工作,令人意外的是,他在转正后不久,所在部门却遭遇了整体裁员。

对于这种情况,我认为它具有两面性。如果你能够接受裁员的不确定性,那么在一些大中小企业或者充满活力的创新部门工作,确实能够加快晋升速度,并且有丰富的学习机会,当然这也伴随着被裁员的风险。相反,如果你选择一个较为稳定的部门,虽然晋升路径可能不那么清晰,但相对而言,工作会更加稳定。对于那些渴望在创新部门工作的人来说,那里更容易取得显著的工作绩效,而且竞争者相对较少。

    \item \textbf{我看你的工作经历大多是TO B的,有接触过TO C的产品经理吗?}

虽然我与TO C端的产品经理接触较少,但即便有观点认为TO B端的年终奖通常高于TO C端,我仍坚信年终奖的多少主要取决于你所从事的是否为公司核心业务,以及公司的整体运营状况。

在谈及工作强度时,我观察到B端的工作节奏确实更为紧张。我们需承担起满足公司全方位需求的重任,而C端则主要服务于部分用户群体。因此,我认为在忙碌程度上,B端的确要更胜一筹。

    \item \textbf{产品经理未来还能有其他岗位的选择吗?比如项目管理之类的?}

目前我也刚入职不久,还没接触到那么多人,因此我看的比较多的是创业和做自由职业者。

    \item \textbf{后期产品经理职业晋升路径和主流发展路线是什么样的呢?有机会成为leader吗?}

    我认为在互联网公司担任领导者是一项很难的事情。首先,现在大幅度涨薪可能性较低,大多数人在跳槽后的薪资涨幅通常被限制在30\%以内,想要实现50\%的薪资增长已经变得相当罕见。其次,有些人可能会选择跳槽到其他互联网公司,而另一部分人可能会投身自由职业,或者重新回到传统行业。虽然互联网行业收入较高,但长时间的工作压力也难免会对身体造成一定的负担。
    
    \item \textbf{在找工作过程中有哪些渠道可以获取相关有效信息呢?}

    (1)关注公众号,可以获取一手信息;

(2)关注一些招聘app,例如牛客网、boss直聘等

(3)还有微博、小红书可以刷一些相关求职社群,比如小红书有一些up主专门发求职信息的,还有一些求职组队的群,去里面了解信息。

    \item \textbf{面试过程中有没有产品经理的面试经验和技巧?}

    最为关键的是自我介绍,务必将自我介绍表述得清晰明了。若个人项目经历较为丰富,需提炼精华,避免冗长。他们会根据你的回答深入询问细节,以验证你是否真正参与其中。若发现过度包装或欺骗行为,面试很可能直接挂掉。

我当时的自我介绍就是围绕我的校园经历,同时巧妙地为面试官设置一些小悬念。例如:我组建了一个团队,共同推进了校巴项目。这时,面试官可能会询问团队构成、项目进展情况以及取得的成果。如何将这些经历讲述得条理清晰?可以借鉴STAR法则,这样面试官会更容易理解。

实际上,在自我介绍之后,面试官心中已大致为你定级。如果交谈初始就感到不合适,那么情况基本不容乐观。如果面试官对你感兴趣,他们会继续提问。

其次会询问一些产品经理的基础概念,例如询问你产品经理的主要职责是什么?你更倾向于B端还是C端产品?并请你阐述选择的原因。
再者,面试官还会根据简历内容评估你的综合能力,如学生干部、社团干部等经历。因此,简历中的每一项内容都需要精心准备,同时要清楚地说明参与这些活动的动机。


    \item \textbf{对于简历中的项目经常会问哪些内容呢?}

    面试中会围绕几个关键问题展开:比如,在项目中你具体做了哪些任务?面临了哪些挑战?又是如何巧妙应对的?如果你能条理清晰地回答这些问题,那么面试基本成功了一半。

项目规模大小并不是关键,重要的是我们在阐述项目时,重要的是能否把项目细节讲述得明白。因此,我们鼓励大家真诚交流,避免过度包装。

记得我刚开始面试时,表现得有些过于刻意,但随着面试到后期,我变得更加轻松。通常,越是较为轻松的装填,面试效果反而越好。
面试并非一场刻板的考试,它更看重的是你与面试官之间的互动,以及你展现出的精神风貌和个性特质。即便你的回答不够完美,但只要你能以饱满的热情去表达,往往也能打动面试官,从而获得通过的机会。

    \item \textbf{产品经理是不是有计算机背景能更有优势呢?}

    拥有计算机背景的人在从事产品方面确实具备一定优势,因为他们能够更加清晰地梳理整个产品链路。或许,计算机专业出身的人士在B端产品领域更为适应,而对于C端产品,则可能需要更多的感性认知。当然,最关键的是,无论背景如何,都必须怀揣对产品的热爱之心。
    
    \item \textbf{你是如何进行offer选择的呢?}

    在选择就业时,我首先考虑的是公司的文化氛围、薪资待遇、地理位置以及业务的可靠程度。值得注意的是,就业的稳定性与业务的可靠性密切相关。若公司从事的是边缘性或试验性业务,那么裁员的风险相对较高,这不仅针对产品经理岗位,对所有岗位而言都是如此。因此,选择核心业务领域的工作更为稳妥。
    
    \item \textbf{对于在校生想从事产品经理这个行业,要去做哪些准备呢?}

    若你缺乏产品经理的相关背景但希望投身此行业,可以做以下四点准备: 
    
(1)首先先了解产品经理基础知识和工作的主要内容。 

(2)尝试组建一个团队来实施项目。 

(3)可以先找家创业的公司拉团队去做项目。 

(4)争取去大厂做产品经理的实习。

许多人会在简历中列举自己的实习经历,但在面试过程中,面试官最看重的是你的创业经历或带领团队完成项目的经验。产品经理的核心能力在于能够凝聚团队,引领大家共同完成任务,并且拥有创业背景往往会更受面试官青睐。
同时在面试时,面试官往往不会过分考查产品经理的专业技能,而是更注重你的软性实力,比如沟通能力、逻辑思维以及创新思维等。


    \item \textbf{产品经理的笔试都需要准备哪些呢?}

    笔试主要涵盖一系列综合能力评估,网上提供了丰富的题库资源,自行多加练习便能应对自如,这些内容类似于公务员考试的行政职业能力测试和申论等。此外,群面环节也相当关键。一些公司若采取海选方式,便会安排群面筛选;反之,若不进行海选,则会直接进入面试环节。
    
    \item \textbf{产品经理会在意你的专业能力或者在校成绩吗?}

    都不会,主要会在意面试时的软技能。
    
    \item \textbf{在你的视角下,产品经理这个行业未来发展趋势是什么?会不会越来越吃香?在AI影响下,产品经理行业会如何变化呢?}

    首先,我坚信产品经理这一岗位不会消失,它会长期存在,并且会有持续的人才需求。尽管当前观点普遍认为互联网行业的增速可能不及以往,但除了互联网公司,汽车制造商、硬件生产商乃至一些传统制造业,对产品经理的需求依然很多。

我认为,AI的兴起会对产品经理产生一定影响。AI技术能够显著提升产品经理的工作效率,并助力打造更具价值的产品。然而,从短期到长期来看,AI还无法完全取代产品经理的角色。AI更多地是从海量数据中筛选信息,而人类则更渴望创新,AI在激发创新思维方面与人类相比还是有较大差距。

    \item \textbf{身边有认识做AI方面的产品经理吗?主要做哪些内容呢?}

    身边做AI产品的同学多数致力于AI技术的实际应用开发。例如,当您在钉钉聊天窗口输入文字时,他们能够提供文字生成服务或优化服务,以帮助您更高效地表达。此外,若您需要绑定日程,系统也会自动为您完成设置。他们的工作更多的是在现有产品中融入AI技术,丰富其功能。
    
    \item \textbf{对于投递简历或者找工作有什么心得吗?}

    (1)多参加招聘会和宣讲会。无论是本科还是硕士,低年级还是高年级,都建议多参加招聘会和宣讲会,积极了解市场行情,这样你才能更清楚地认识到工作的实际情况,并找到适合自己的位置。

(2)多关注感兴趣公司的公众号或直播。我曾在关注某大厂公众号时,发现了一个模拟面试的直播,而直播主讲人恰好成为了我的面试官。因此我对他的面试风格有了一定的了解,面试过程中也很顺利。

(3)投简历要保持一定的投递节奏。建议每周投递15-20家公司的岗位,第一周投递,第二周进行笔试,如果顺利,第三周即可进入面试环节。

(4)养成记录的习惯。投递简历时,建议使用飞书文档记录所投递的公司、岗位及进展情况,以便随时掌握求职进度。

(5)即使你已经入职某家公司,若收到其他公司的offer,也可以尝试谈判薪资。毕竟,HR招聘人才不易,且招聘成本较高。

(6)想找暑期实习要尽快准备。对于互联网大厂而言,暑期实习至关重要。尤其是本科三年级到四年级的暑期实习,如果表现良好,就有机会获得转正的offer。这样一来,秋季招聘时心理压力就会大大减轻。因此,从3月份起就要开始在网上投递。


    \item \textbf{如果再重来一次校招,你会做出哪些改变?}

    若能重新选择,我倾向于投身开发领域,不再涉足产品工作了。产品经理的工作确实让人心力交瘁,事务繁杂,且难以拥有自己的私人时间。所以,如果有机会重来,我愿意尝试走开发的道路。哈哈。
    
    \item \textbf{对将要找工作的学弟学妹们有什么建议?}

    (1)无论你想做什么工作,务必做好充分的预先准备。
    
(2)积极寻求实习机会。例如,在大学期间就应参与实习等活动,不要等到毕业之后才开始。首先,实习能让你积累宝贵的工作经验,未来求职时更加得心应手;其次,通过体验不同公司的不同岗位,你将更清晰地了解自己究竟喜爱何种工作。

(3)广泛搜集信息,多与人交流。特别是与学长学姐、在职人员交谈,从他们的经验中,你可以预见自己的职业发展路径,了解该行业和领域未来的发展趋势。

这里分享一个小窍门:如果你有意向在某个公司实习,但又担心无法转正,同时又希望留在该公司,你可以利用公司内部的通讯工具,主动向其他同事了解不同部门是否有适合的机会。特别是对于非技术岗位的同学来说,更应该积极一些。

\end{itemize}
