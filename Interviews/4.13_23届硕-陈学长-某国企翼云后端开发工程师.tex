\newpage
\section{23届硕-陈学长-某国企翼云后端开发工程师}
\begin{itemize}

\setlength{\parindent}{2em} 
    \item \textbf{可以介绍下您毕业后的工作经历吗?}

23年6月从复旦毕业之后,进入了一家国企(非传统类型)从事后端开发相关的工作。

    \item \textbf{目前薪资待遇和工作强度是怎么样的呢?}

待遇:当时拿到的是sp offer,总包大概对应大厂的白菜水平。
工作强度:不同部门差异较大,我这边属于比较卷的组了,早上是9点打卡,我个人来说是一周大概2-3天加班到9点之后,其他时间例如周五大概是六七点跑路。总体来说略低于互联网平均强度。


    \item \textbf{您当初选择这份工作,从事这个行业的理由是什么呢?}

当时的想法是待遇还行的情况下求稳吧,然后也能积累一定经验。

  
    \item \textbf{这个工作/行业有哪些最令人满意的地方?}

人际关系非常简单,组里氛围还不错,虚头巴脑的东西比较少,把活干好就行;虽然是国企,但是技术上还算比较前沿,可以学到不少东西。
(相比较互联网许多裁应届生的公司)比较稳定,合同期内还没出现过裁员现象。


    \item \textbf{这个工作/行业有哪些最想吐槽的地方?}

有着国企的部分通病(一些不是非常严重的形式主义、官僚主义)。
食堂比较贵,味道一般。
跳槽认可度一般(中厂水平)。
软件基建不太好(影响开发体验)。

    \item \textbf{这份工作/行业带给您最深感受/影响是什么?}

要想升级快就要用力肝(卷工时卷产出)。
几乎每个人都有不同的焦虑,担心年纪大了失业,担心没时间陪伴家人,担心卷不动被淘汰,不过这也是这个时代各个行业的通病吧哈哈哈,这一行起码钱还行咯。

    \item \textbf{在你的行业中,职业晋升的通常路径是什么?有哪些职业发展的方向、机会或障碍?}

想晋升:
卷(让领导看见和认同的情况)
跳槽(积累技术,找到匹配的方向)
障碍:
方向很重要,跳槽现在比较讲究技术栈和经历匹配,所以最好不要选择过于小众的方向,积极向风浪靠拢,比如 ai 相关。

    \item \textbf{行业近年来的主要发展趋势是什么?您预测未来几年内,(在政策/AI技术等因素的影响下)行业将会如何变化呢?}

说实话我没能力预测。猜测的话,内部在强调内循环经济的背景下
可能政策会靠拢智能驾驶、新能源、低空经济、国企信息化改革等。外部的话AI浪潮这么猛,相关基础设施比如云计算,相关应用比如智能驾驶,AI医疗,特定领域大模型等交叉方向可能比较有前景。

    \item \textbf{对于计算机专业的在校生,如果将来想要从事这个行业,找到这样的工作,需要做哪些准备呢?}

实习经历 >= 计算机基础 >= 相关竞赛(不同公司标准不一)
算法方向的话论文也是非常重要。

    \item \textbf{您在找工作过程中通过哪些渠道获取相关就业信息呢?}

各大公司官方公众号、牛客网、BOSS、往届内推

    \item \textbf{可以复盘下您在校招时的经历和心得吗,如果能重来一次校招您会做哪些改变呢?}

毕业和招聘卡一起了,能顺利度过就非常不错了,不想再来一次了哈哈哈。
总体原则:越早准备越好,多了解行业情况,多看面经,多思考多问(什么发力过早会累这种鬼话不要信)。

    \item \textbf{对那些迷茫于找工作的在校学弟妹们,您有什么寄语吗?}

千万不要有太大压力,最不济也就是暂时找不到工作,也没啥大不了的,因为找到工作的人也不知道能干多久,因为现在看似高薪光鲜的工作几年后可能啥也不是哈哈哈,保留一个应届生身份,还有很多方向可以去探索(考公、选调、人才引进、各种单位的信息部门,甚至是转行,以考复旦的难度来看,拿出你们的五成功力就ok啦)。只要不彻底摆烂,在人生的维度上,毕业季这点事只能算一个小波澜哈哈哈。

\end{itemize}