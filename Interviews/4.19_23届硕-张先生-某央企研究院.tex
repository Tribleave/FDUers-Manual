\newpage
\section{23届硕-张先生-某央企研究院}
\begin{itemize}

\setlength{\parindent}{2em} 
    \item \textbf{可以介绍下您毕业后的工作经历吗?}

张学长在2023年6月硕士毕业后,直接进入了一家央企工作。他在研究生期间的主要研究方向是后端开发相关技术,但在研究生后期,尤其是实习期间,他转向了云计算领域,尤其是在字节跳动的实习经历中进一步加深了对云计算的理解。进入央企后,他的工作重心逐渐转向与AI相关的项目开发,涉及从传统的小模型算法到近年来大模型的应用和开发。张学长目前在一家央企的研究院AI部门工作已满一年,主要负责AI技术的开发与应用,尤其是大模型在实际业务中的落地。他提到,尽管进入AI部门是意料之外的安排,但通过这一经历,他逐渐适应了从后端开发到AI的转型,并在新的技术领域中找到自己的定位。

    \item \textbf{目前薪资待遇和工作强度是怎么样的呢?}

张学长提到,央企的薪资待遇相比互联网大厂低,达到互联网行业平均水平。他坦诚地指出,虽然不能与互联网头部大厂的高薪相比,但薪资仍然属于可以接受的范围。此外,他提到工作强度相对较低,每天的工作时间大约从早上九点到晚上七八点,每周工作五天。工作节奏较为稳定,紧迫感不强,这种安排让他有更多的私人时间。张学长强调,这种较低的工作强度和相对灵活的时间安排,是他选择央企的一个重要原因,因为这让他能够投入更多时间到个人感兴趣的开源项目中,发展自己的第二职业曲线。


    \item \textbf{您当初选择这份工作,从事这个行业的理由是什么呢?}

(1)工作与生活的平衡:在字节跳动实习期间,张学长感受到互联网大厂的工作强度非常高,私人时间被大幅压缩。他更倾向于在工作和个人生活之间找到一个平衡,因此他选择了工作压力较小、工作时间更为灵活的央企。央企的相对自由的工作安排使他能够保留更多的私人时间,这对他参与开源项目和自我提升非常重要。

(2)长远发展考虑:张学长提到,他在字节跳动实习时接触的是云计算基础架构部门的工作,他意识到基础架构领域更适合由国家层面来主导和推动,这样更符合国家战略的需求。因此,他选择了央企来从事这方面的工作,认为在央企可以更好地发挥自己的专业特长,同时也能为国家的云计算和AI发展做出贡献。

(3)稳定性与职业安全:在当时的就业形势下,互联网行业开始出现波动,张学长出于对未来职业稳定性的考虑,选择了相对稳定的央企。尽管央企的薪资不如大厂高,但提供了更多的职业安全感和发展空间,尤其是在经济形势不稳定的情况下,央企的长期职业规划更为可靠。
  
    \item \textbf{这个工作/行业有哪些最令人满意的地方?}

(1)工作时间的灵活性:他提到,周末没有人打扰,这让他可以有自己的私人时间,这一点对他来说非常重要。张学长热衷于参与开源项目,他希望能够有足够的时间来探索自己感兴趣的领域,并将这作为自己职业发展的第二曲线。这种时间上的灵活性,使他能够在工作之外保持对技术的热情和探索。

(2)工作氛围:他对公司内部的工作氛围表示满意,尤其是与同事之间的相处。张学长提到,他所在的公司是一家研发类型的企业,这种环境与他在字节实习时的氛围非常相似。他强调,尽管公司性质不同,但研发人员之间的互动和合作精神依然非常好,这让他在工作中感到舒适和愉快。

(3)工作内容的实际应用:张学长更喜欢能够实际落地的项目,他提到在工作中看到自己开发的系统和算法被应用于现实场景,能够为客户解决实际问题,这带来了强烈的成就感。他认为,相比在学校里做纯粹的学术研究,能够看到自己的工作产生实际影响,更加有意义。


    \item \textbf{这个工作/行业有哪些最想吐槽的地方?}

张学长提到,作为央企的一员,他提到公司内的流程过于复杂,相对于互联网公司,央企的流程更加繁琐,这会影响工作效率。他指出,这是在安全与效率方面的取舍。

    \item \textbf{这份工作/行业带给您最深感受/影响是什么?}

张学长在工作中得到的最深感受是真正的稳定感只能来源于自身。他一开始选择这个工作是因为考虑到行业形势变化,希望找到一份相对稳定的工作,但实际工作中他认识到,没有绝对的稳定,只有通过提升自己的能力才能获得真正的安全感。此外,他还提到他在尝试进行开源项目时发现,这方面的工作比想象中更难,需要投入大量的时间和精力来从零开始理解并掌握新的领域。

    \item \textbf{在你的行业中,职业晋升的通常路径是什么?有哪些职业发展的方向、机会或障碍?}

张学长提到,职业晋升的关键在于贡献和主动性。他解释说,如果你在项目中有较大的贡献,或者在解决项目中的难题时表现突出,你甚至可以跳级晋升。晋升的路径通常是从项目组长开始,之后可能晋升为总监,负责更大范围的项目和团队。要在行业内晋升,除了做出贡献,还需要对自己有较高的要求,并且保持较强的主动性去承担更多的责任。

他也提到,虽然央企在职业稳定性方面相对较好,但依然存在类似私企的“中年危机”。在经济形势不佳时,央企也会有裁员的可能性。因此,他建议要持续提升自己的业务能力,并以创业者的心态去看待自己的工作,这样才能在职业生涯中保持竞争力。

    \item \textbf{行业近年来的主要发展趋势是什么?您预测未来几年内,(在政策/AI技术等因素的影响下)行业将会如何变化呢?}

张学长提到,近年来国内的主要发展趋势是阿里云的牵头下,中小企业逐渐开始上云。阿里云自身因为有淘宝这样的巨大需求,所以在云计算领域做得非常强。对于中小企业来说,自己购买和维护服务器的成本很高,通过将服务部署到云上,可以大幅降低运营成本。此外,他还提到,虽然中国的互联网消费已经非常发达,但工业互联网化的程度依然很低,这表明中国的工业企业在数字化、自动化方面还有很大的发展空间,未来的潜力巨大。

在未来几年,张学长预测,随着大模型的兴起,算力将成为一种重要的战略资源,类似于过去通信技术的重要性。特别是对一些中小型创业者来说,由于资金限制,他们可能无法购买足够的计算设备,因此通过云平台租赁算力将成为一个非常实际的选择。另一方面,央企在未来的云计算发展中将承担更多的国家战略任务,不仅要关注盈利,还要确保国家信息安全,防范来自国外的网络攻击。

\end{itemize}