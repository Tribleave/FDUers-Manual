\section{其他行业(如金融、智能制造、车企等)}


\subsection{银行总行管培生}
\subsubsection{前言}

大家好呀,这篇文章主要是关于三大政策行和六大国有行总行管培生的求职经验贴,方向是金融科技。最终我也是很幸运的通过了农行总行金融科技管培生和邮储总行管培生的笔试和面试,虽然最后因为自己的一些原因考虑没有选择去,但是也希望将自己的这份经历分享出来,希望可以帮助到大家~
具体可以参考学长的牛客网,还有许多的面经可以参考:\url{https://www.nowcoder.com/discuss/578953210505732096?sourceSSR=users}

\subsubsection{管培生}

在正式开始前,也希望跟大家谈谈什么是管培生以及管培生大概得工作内容,以下主要是基于个人的理解以及根据相关信息得出的,可以做个参考。管培生是管理培训生的简称,这是一个外来术语,是外企里面“以培养公司未来领导者”为主要目标的特殊项目,不仅外企,许多国企和民企也纷纷开招管理培训生。简单来说就是作为未来的管理者进行的人才储备项目,虽然实际的工作中可能基本不太用写代码,但是可以接触到高层决策者,并且可以参与到公司的实际管理运营中。个人觉得是上限很高的一份职业,虽然硬性的薪资比不上互联网大厂,但是各种福利和机会则是互联网完全比不了的。

\subsubsection{三大政策行和六大行}

一般管培生最主要去的是三大政策行和六大行的总行,三大政策行指的是国家开发银行、中国进出口银行和中国农业发展银行,六大行指的是工商银行、农业银行、中国银行、建设银行、邮政储蓄银行和交通银行。无论是哪一家的总行竞争都很激烈,一般来说政策行>六大行,而六大行中又以工商银行、农业银行、中国银行、建设银行为主,其中工商银行被称为“宇宙行”,竞争尤为激烈,你能想象线下面试中一个组十个人有牛津、剑桥、哥大和清华博士的感觉嘛。所以最好先找准自己的定位,分清楚自己的优势所在,选择适合自己的银行投递。(跟神仙们打架真的太累了qwq)

\subsubsection{时间线}

银行开始投递时间相比互联网会晚不少,一般是在九月份才开始,一系列线下笔试和面试可能会一直持续到十一月甚至十二月,所以时间线算是比较长的了。想要准备好相关笔试和面试,信息是十分重要的,这里我觉得主要的几个信息来源是:\textbf{公众号(银行招聘网、银行考试信息汇总等)、官网、校招VX群},一般来说是通过公众号每天推送的消息知道某家银行是否开始投递,然后再在具体的官网上进行投递即可。

\subsubsection{简历}

简历的投递是第一步,但也是十分重要的一步。随着这几年银行和金融行业的热度激增,投递报名的人也是一年比一年多,但整体的hc不仅没随着趋势增长,反而有下降的势头,所以竞争是越来越激烈的。三大政策行和六大行基本都会要求线下笔试和线下面试,但是线下不同于线上,线下是需要提供考场的,所以一定会控制参与考试的人数。那么如何控制呢?那就只能简单通过简历筛选来做了。所以简历和相关信息一定要认真填写,把自己最突出的地方进行展示。

那么如何写好一份面向总行管培生的简历呢?我觉得主要是从以下几个方面入手:

•	领导能力。管培生项目是为了培养公司未来领导者,所以是否有领导力和团队合作能力会是HR进行筛选的重要衡量标准。那么如何在简历里面体现领导能力呢?最主要的便是将自己学生时期所担任过的学生干部以及从事的学生工作写上去,此外,由于是金融科技方向,担任的项目负责人也可以写上去,按照\textbf{担任的职务->负责的工作->培养的能力}进行介绍。不同于互联网行业需要写上技术栈和技术实现,管培生更加看重的是你能否领导团队,有效沟通和协调,解决问题并推动项目的实施,也就是那句话:“船长最重要的不是造船,而是激发水手对大海的向往”。所以需要按照这个思路,调整简历中的侧重点从技术实现难度转移到团队合作和团队管理上。

•	学习能力。因为管培生可能会进行轮岗,会在不同部门进行历练,所以学习能力最好也要能够在简历中进行体现。最直观的方式就是绩点和排名,可以将成绩不错的科目写上。此外,积极参与不同的项目和学生工作也可以算作一种学习能力的展现。

•	解决问题能力。学生和工作是两种完全不同的阶段,学生时期最主要的就是学习好对应科目即可,学习方法和相关策略是比较固定的,但是工作的时候就完全不一样了。工作时候会遇到各种各样的问题,而这些问题往往并不一定有标准答案,有些时候甚至连问题都不一定能够明确。所以,HR在筛选简历的时候也会比较看重这方面,看能否能够有解决复杂问题的思路和能力。我认为解决问题一般是这几个步骤:\textbf{发现问题->定义问题->解决问题->反思总结},相应的,简历中也应该按照这个思路进行体现。比如项目中遇到了什么问题,怎么思考解决方案的,如何落地解决的,最后从这件事情中学到了什么

\subsubsection{笔试}

银行总行考核包含笔试和面试,这两个一般都是线下进行的,笔试通过后才会安排面试。

先说笔试,笔试包含行测、英语、专业知识、经济金融知识、时政以及银行特色知识,政策行还会考核申论。我是大概准备了一个多月,个人觉得刷题是尤为重要的,尤其是在时间不太够的情况下。除非时间比较充足,不然一般不是很建议看网课,看纸质版材料学习 ,然后多刷题就好了。以下是各个考试内容我的准备方式:

•	行测、经济金融知识、银行特色知识:《银行招聘考试一本通》、粉笔app、北森题库。这三者中尤其是行测为重点,因为行测占比是最大的。此外,根据我去年线下笔试的经验,笔试的时间一般都是很不够的,很考验对于行测的技巧解法,所以一定要多刷刷行测题目,掐点做。在考试的时候,对于一些看一眼没有思路的题目,最好立即跳过,等有时间后再回头解决,因为平均一道题的时间就一分钟,所以一定要有所取舍。最好先通过《银行招聘考试一本通》这本书进行学习,然后刷往年真题,有时间可以掐点在粉笔app上刷题,北森题库也可以作为参考。经济金融知识和银行特色知识在临考前熟悉熟悉常考题目就好,做不来放平心态放弃就好,一般金融科技方向的同学也不太会这些题目,所以不用太担心。

•	专业知识(计算机):牛客、粉笔app。专业知识不同银行考察难度不一样,有些就简单几道选择题,但是像农行还会有道编程题。对于专业知识部分,可以在牛客网和粉笔app上进行刷题,不过只用针对一些常见题目有一定熟悉就好,不用针对大范围刷题,性价比不高。

•	英语:每天1~2篇阅读理解。笔试也会针对英语进行考察,英语的话专门花太多时间同样性价比不高,保持每天背一背常见单词,每天做1~2篇阅读理解,保持语感和做题速度即可。

•	时政:银行考试相关公众号。时政也不太需要专门去准备,因为这个考察的分数并不多,而且考察的范围看会很广,也就是在考试前在银行考试相关公众号上临时看一看就好。

\subsubsection{面试}

笔试通过后会进行面试,除了极少数比如交通银行总行一面是线上外,一般来说都是线下参加。参加线下面试都是在总行,基本都在北京,参加的时候需要穿着正装。线下面试一般包含半结构化面试和无领导面试。

半结构化面试是指面试构成要素中有的内容作统一的要求,有的内容则不作统一的规定,也就是在预先设计好的试题的基础上,考官向应试者又提出一些随机性的问题。面试流程一般由面试者进入考场先进行自我介绍,接着考官会根据介绍和简历上的内容随机提问问题,自我介绍的时间多为1-3分钟。一般有两种形式:一种是逐一面试,另一种是多人面试,银行半结构化面试的考察重点是自我介绍、个性特征、行为经历和求职动机。

不过也会有针对专业知识的提问,比如农行总行的面试是要求设计银行网站,具体要考虑分布式、高可用以及高并发,是最接近互联网大厂面试的一次。其他银行的半结构化面试基本比较常规,包括但不限于考察个人经历、团队合作能力、人际沟通技能、解决问题的能力以及组织活动能力等。

一般来说,为了准备半结构化面试需要准备面试常见问题,主要是根据自己的经验和技能,思考有哪些地方可能是面试官可能会问到的,对此进行一些准备。此外,针对一些常见的半结构化面试问题,也需要大致有一些自己的想法,一般来说需要积极向上,寻求团队合作以及迎难而上的特点。

无领导面试指的是面试者需要在没有明确领导或指导者的情况下解决给出的讨论问题、一般来说,需要先个人单独先对问题进行阐述,分析讲解自己的想法,然后开始进行小组讨论,得出小组内部一致的想法和结论。此外,也会有辩论形式,将一个组的同学分为两个队,每个队会安排相反的论点,尝试说服对方,但是最后两队需要达成一致。

在这个过程中需要展示自己的领导能力、自主性和解决问题的能力。具体来说,为了更好地展现自己,我觉得需要做到或者展现出以下几点:

理解面试要求:首先最重要的便是要确保已经正确无误的理解了面试题目的要求,比如辩论题目最后两队是否需要达成一致。我第一次参加无领导小组面试的时候就遇到对方队一直到最后还在和我们辩论的情况,最终都没有满足题目达成一致的要求,所以得分偏低。

•	自主性:在面试中展现出积极主动的态度,提出解决问题的方法和建议。展示能够自我激励、自我管理并在压力下做出决策的能力。

•	领导力:在面试中展现出领导能力,包括激励团队、制定目标和战略,并展示如何带领团队实现目标。但是也不一定就要做领导者,积极推动讨论进行下去,或者给出一些有用的想法也是很不错的。

•	团队合作:无领导面试中涉及到团队合作,需要展现出良好的团队合作和沟通能力。要善于倾听他人意见、激励团队并寻求共识。


\subsection{银行数据中心}
\subsubsection{简历投递}

首先从岗位选择上,研发中心和数据中心是可以同时投递的,但要注意的一点是投递两个岗位的简历最好各有侧重,不要完全一样。投递研发中心,简历要着重突出自己的开发能力。投递数据中心,简历要偏向运维、数据安全,当然也可以表现自己的开发能力。

这里说一下研发中心和数据中心的区别。从工作内容角度,研发中心就是通常意义上的软件开发,适合喜欢写代码的同学。数据中心则是投产运维,研发中心很多事情要数据中心批准后才能执行。从子公司化的角度,数据中心由于数据的隐私性和安全性,是永远不可能被子公司化的,数据中心本身不具备盈利能力,但又必须存在,所以会一直作为直属机构绑定总行。而研发中心是能接项目开发软件盈利的(作为银行的内包),就有被子公司化的可能。

一般而言,数据中心的选择优先级是高于研发中心的,但也要因人而异。第一,从工作时间上,数据中心因为投产要值夜班调休。每半个月通宵一次,有时周六要去上班投产,虽说值班可以调休,但要提申请领导批准,所以实际上调休并没有落到实处。第二,数据中心一般会位于城市偏僻的地方,以后的生活通勤能否接受需要考虑。第三,数据中心的岗位间差异大,闲的岗位很闲,忙的岗位值班会很多,而岗位是入职后再确定的,所以具有很大的不确定。第四,数据中心不适合喜欢写代码的同学。

\subsubsection{笔试}
银行的笔试基本不会淘汰人,但还是要准备一下。基本构成为:行测+英语+计算机+其他+银行文化。

银行文化考前一天背就行,计算机主要是数据结构、计算机网络、计算机组成原理、操作系统、数据库设计、sql语句、程序代码填空。如果准备时间充足,可以上牛客网刷一下计算机的题库。如果时间不足,就刷一下银行往年的题。

行测和英语无需特殊准备,正常答即可。行测部分,数量计算可以放在最后答,答不完就蒙,都选一个答案。英语部分,阅读理解题的分值较高,可以选择优先做阅读理解。其他部分视银行而定,有的银行各岗位间不分卷,就会考察大量的经济管理题,这种就凭感觉做了。

\subsubsection{面试}
一般在面试前会要求做心理测评或AI测评,一定要按时完成,否则流程就会终止。有的银行发的测评通知会被邮箱标定为垃圾邮件,所以如果没有按时收到记得去垃圾邮件里找一找。

数据中心、研发中心的面试都是技术面。八股是一定会问的,银行问的是java八股,且较为基础。也有结合自我介绍延展的八股,数据中心也会问网络安全、数据库、密码学、区块链的八股。

项目经历也会问,如果有银行相关的经历会问的更细一些,但总体上不难。有的会问知不知道国家对银行的政策,这个问题答信创和数字化转型就好。还会问一些性格方面的东西,表现出认真负责、团结协作就好。如果考过软考的证,还会问怎么考的证。

\subsubsection{签约}
面完大概一周会通知体检,但是差额体检,也就是说即使体检了也不一定能拿到offer,交完体检报告后几天会通知签约会。签约会当面谈薪资福利,不签就可以离场了,否则当场签三方,这一点一定要慎重,因为银行发offer的时间在国企里算早的,后续很大概率会毁约,要衡量一下毁约费是否能担负起(以农行为例,违约金是2万),如果考上公务员是不用赔违约金的。最后就是,即使没有通知到体检或体检后候补了也没有关系,到年末会有一波补录,同样有机会拿到offer。


% \chapter{求职过程}

% \section{求职经验(从准备,到写简历,到笔面试,再到选offer全流程)}
% \subsection{前端}
% \subsubsection{toB, toC, 面试,实习}
% \subsection{测试/运维}
% \subsection{彭佳汉(后台JAVA开发)}
% \subsection{产品(刘济尘)}
% \subsection{算法}


% \subsection{国企(彭佳汉)}
% \subsection{公务员和事业单位}
% \section{宣讲会(作用,信息获取,如何利用)}
% \section{简历}

% \subsection{曲yue}
% 简历一般包括:个人背景介绍、个人技能(优势)、工作(实)经历、项目经历、所获奖项和自我评价。
% \begin{itemize}
%     \item 个人背景: 基本信息、学历、意向岗位、目前所在城市、预期到岗时间等。
%     \item 个人技能: 分点论述自身的职业技能,相关度高的技能可以写到前面,兴趣爱好可以挑个人擅长且与岗位相关的来写,用括号进一步描述。
%     \item 工作(实习)经历: 首先要逻辑清晰,总结自身的工作内容,分点论述,按照重要程度和连贯性依次排列。其次是数据量化,数字化的描述会让表达更加准确、更有冲击力和说服力。最后是行为动词,行为动词能够直接、简介的体现出个人的能力以及在团队中的作用,比如:搭建、统筹、制定、协调、把控等等。
%     \item 项目经历: 项目经历不需要全部写到文档中,只需要挑选1-2个含金量高、知名度高的项目写到简历中,且重点突出介绍项目内容、个人在项目中担任的角色、项目开展过程、项目结果等。注意要分条理和逻辑的书写。
%     \item 所获奖项: 可以将重量级的奖项放到最前面,其他奖项放后面。
%     \item 自我评价: 首先表明你的工作经验和所处行业,并突出你的成果以及对公司业务的贡献,最好用数据量化体现出来。其次突出你的专业能力,是否有带团队的经历和相关资源。最后补充自己的性格,沟通能力、表达能力和学习能力都如何。
% \end{itemize}
% 投递简历节奏:
% 投简历要保持一定的投递节奏。建议每周投递15-20家公司的岗位,第一周投递,第二周进行笔试,如果顺利,第三周即可进入面试环节。
% 2.2.3记录简历过程
% 投递简历时,建议使用飞书文档记录所投递的公司、岗位及进展情况,以便随时掌握求职进度。

% \section{投递策略}
% \subsection{曲yue}
% 1.投递简历时间:建议工作日早上9:00或者下午2:00,这样hr在上班时第一时间能看到。


% 2.修改投简历的"招呼语"例如找开发工作,写:2年互联网开发经验,独立负责过xx项目。这样既表明了经验,又能增加hr的印象。


% 3.一家公司重复投:一家公司有多个hr,一个没回复,可以联系其他人投简历。


% 4.勤沟通:有时HR看不到你的简历不要不好意思给他发消息,可以再次提醒他一下。


% 5.擅长使用工具:
% 例如可以使用简历模板来方便排版、或者使用AI工具辅助修改简历等。


% \section{笔试(线上/线下/八股/算法题/行测/申论/英语/性格测试)(笔试)}

% 考试类型,出现场景,备考经验

% \section{面试(AI面/群面/无领导小组讨论/技术单面/结构化单面/英语面/HR面/主管面)(王世聪)}




% 我们在校招期间会面临的面试大体可以分为两大类:非技术类和技术类。非技术类的面试包括泛体制内的结构化面试、无领导小组讨论/辩论赛、普通群面(多对一)、单面(如企业中的HR面、主管面等)。技术类的面试则涵盖算法题、项目考察、技术基础知识(八股文)、业务场景题等。


% \subsection{结构化面试}


%  结构化面试通常出现在公务员选调及一些国企事业单位的面试中。这类面试通常给出一个开放性的问题,提供一定的思考时间,让应试者展开回答。题型通常包括组织规划类、观点明晰类、人际交往类等。



% 针对结构化面试的应试策略可以分为两种:


% 一种我称之为\textbf{素材梳理式},适合绝大多数没有接触过行政话语的同学,因为按照往年各机构整理出的参考答案,可以发现,其实每一类题型都可以整理出对应的模板,里面嵌入的内容也有共性,比如组织规划类,经常出现事前要充分调研,事后要分析反馈,事中要维持秩序,做好应急预案等,那这些一个个通用的举措,就是我们要积累(背下来)的素材,再根据具体的题目情景进行细化的表述,就形成了一个各个环节紧密耦合,非常细致且全面的回答。这种方式需要通过大量的针对例题的实战练习,逐渐打磨出一套适合自己的答题模板,并积累内容素材和套话。练的特别熟悉之后,在正式面试时,通过快速反应、流畅表达和全面的内容输出给面试官留下深刻印象。这种方式的重点是练习时要掐时间脱稿,尽可能模拟面试现场情景,以克服紧张情绪,改掉不良口癖和动作,找到适合自己的表达方式。这样准备面试的同学,可以关注复旦大学的基层就业协会,它们每年都会在秋招期组建各个省的复旦官方选调群,复试期间里面会有复旦的同学们一起约线下对练,可以找一些同省考试的同学一起面对面练习,这样能听到对方给你的反馈意见,相互纠错。


% % 另外一种我称之为\textbf{观点爆破式},适合平时就非常关注时政议题,有大量的观点累积,同时具有一定的思辨和表达能力的同学。这种方式需要我们突出重点,输出一个漂亮的观点,展示我们思考的深度。试想一下,在面你之前面试官可能都听了一天的模板,那些相同的套话和素材可能都听麻了,而这个时候你出场,从一个新颖的角度,细致阐述了一个有思考的论点,就会让面试官眼前一亮,脱颖而出。还用组织规划类的题型举例,这种策略的叙述格式类似于:虽然组织这个活动需要事前调研,准备,事后分析等等诸多环节,但是在这个活动里最关键/核心问题/最需要把握的环节其实是(比如说要做好紧急预案),因为这个活动具有某些鲜明的特点(比如大夏天在户外容易中暑啊,或者人员密集聚集容易踩踏,或者是线上直播容易有舆情之类的),曾经就有类似的活动因为没有注重这类隐患而出现了严重的问题(举例子),所以我们该吸取前车之鉴,做好以下工作(阐述举措)。所以梳理式是熟练有结构的罗列素材(类似BFS),爆破式是针对一个环节/观点进行深入阐述(类似DFS),这种准备的话需要同学们关注对应省份近一年来的重要议题,想一些自己的观点,并且记录下来,有新想法,或者在网上遇到了相关的案例,就把之前的记录更新,这样就能逐渐积累出自己的观点和案例库,增加现场面试时输出观点的机会。



% 另外一种我称之为\textbf{观点爆破式},适合平时关注时政议题、拥有大量观点积累并具有一定思辨和表达能力的同学。这种方式需要我们突出重点,输出一个漂亮的观点,展示我们思考的深度。
% 试想一下,在面你之前面试官可能都听了一天的模板,那些相同的套话和素材可能都听麻了,而这个时候你出场,从一个新颖的角度,细致阐述了一个有思考的论点,就会让面试官眼前一亮。
% 例如,回答组织规划类题型时,可以从活动的关键环节入手,指出最需要关注的地方(比如说虽然组织这个活动需要xxx,xxx等诸多环节,但是在这个活动里最关键/最核心问题/最需要把握的环节其实是要做好紧急预案),然后分析活动特点(如户外活动容易中暑、人员密集容易踩踏、线上直播易引发舆情等),并结合实例进行说明(比如曾经就有类似的活动因为没有注重这类隐患而出现了严重的问题),接下来再介绍举措并解释理由,最后几句话带过其他不重要的环节。
% 这类策略在准备上要求同学们关注报考省份近一年来的相关议题,形成自己的观点,并且记录下来。之后有新想法,或者在网上遇到了相关的案例,就不断更新之前的观点和案例库,以备面试时灵活应对。


% 总的来说,梳理式类似广度优先搜索(BFS),强调熟练的、完整的且结构化的素材罗列,可以短时间内速成;爆破式则类似深度优先搜索(DFS),着重深入阐述某一环节或观点,需要一些基础能力和日常积累。根据自身特点选择合适的应试策略,才能在面试中脱颖而出。

% \subsection{无领导小组讨论/辩论赛}






% 无领导小组讨论/辩论赛的面试常出现在国企和部分省份的选调,一些大厂的非技术岗也有这种形式的面试,有些还混合着来,讨论中间掺杂辩论环节,最后再进行总结,这种我们其实在校园里有过体验,求职面临这样的面试有几点经验:


% 1.切记克制情绪。这种面试不是为了吵赢其他的面试者,而是要把好的一面展示给旁听者,在和其他面试者们一起讨论甚至辩论的过程中肯定会有不同意见的碰撞,保持耐心倾听和礼貌表达,注意千万不要带着情绪,表现得过于强势,你一上头,旁观者就很下头,结果就是被挂。

% 2.注意遵守规则。很多学长姐反馈的经验中都提到,如果自己的发言环节有时间限制,一定不要超时,尤其是在一些选调面试中。如果时间到了,就算没说完也要停下来,避免超时。

% 3.争取有效输出
% 有效输出是指能够给面试官留下深刻好印象的表现。在无领导小组讨论中,虽然会有领导者(leader)、计时员(timekeeper)等角色,但校招面试中,面试官通常不会特别在意这些角色标签。一场面试下来,讨论的质量可能并不高。在这种情况下,有两个动作会特别出彩:

% (1)总结:这是最有用的动作,总结也包括两部分,一个是面试的最后会要求选出一个人做总结汇报,这是一个备受关注的环节。如果能够争取到做总结的机会,并做好总结,会非常加分。即使无法担任最后的总结人,也要主动推举一个合适的人选,并给出充分的理由,避免在这个环节变成小透明。

% 另一个重要的部分是进行阶段性总结。当讨论变得混乱、大家提出许多不同角度的阐述时时,能够及时对内容进行概括和归纳,并向其他人确认,是非常有价值的。这种阶段性总结不仅可以使讨论更加清晰明了,还能引导大家进入下一个讨论环节。当面试官对讨论内容感到困惑时,他们往往会期待有人能够清晰地总结当前的讨论进展。这样的总结会让你很自然的成为推动讨论进程的人,并有效地掌控讨论的节奏。





% (2)观点靠近业务:提出与应聘单位和岗位的业务内容相关的观点,能给面试官留下深刻印象。在面试前,多准备一些关于应聘单位和岗位的信息。在表达观点时,尽量结合这些要素,让人感受到你对这个单位或者这个岗位的充分了解和重视,会有意想不到的收获哦。
% \subsection{技术面}

% \subsection{英语面}

% 面试总结成n大类

% 目的,形式,技巧
% 1.大厂技术类(算法,开发,八股?code)
% 2.非技术单面(公务员结构化面试(梳理式,爆破式),HR/主管面,国企面,产品等非技术岗的面试)
% 3.群面(正常群面,无领导小组讨论,辩论)
% 4.英语面(外企水面,)





% \section{体检}
% \section{背调与政审}
% \section{心态调节}
% \section{创业}
% \chapter{后续过程}
% \section{offer信息判断(咨询途径/询问方式/多方验证)}
% \section{offer比较与选择(管培/轮岗/起薪/福利/薪资组成/考核机制/裁员风险)}
% \section{签约(两方/三方/应对逼签)}
% \section{毁约(流程/规则/违约金)}
% \section{入职}
