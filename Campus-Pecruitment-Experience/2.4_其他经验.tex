\section{其他经验}

\subsection{关于非技术类面试}

% \section{面试(AI面/群面/无领导小组讨论/技术单面/结构化单面/英语面/HR面/主管面)(王世聪)}




我们在校招期间会面临的面试大体可以分为两大类:非技术类和技术类。在泛体制类单位的面试中,不会有大厂那样很刨根问底的技术面试,通常问问简历项目和一些基础问题就结束,甚至有的都干脆不问,更重要的是一些非技术类面试。我定义的非技术类的面试包括泛体制内的结构化面试、无领导小组讨论/辩论赛、普通群面(多对一)、单面(如企业中的HR面、主管面等)。技术类的面试则涵盖算法题、项目考察、技术基础知识(八股文)、业务场景题等。


% 面试归根结底是人和人的互动,尤其是非技术面,表达诚意,展示热情,可能比回答的内容更重要,相较于把它当成一次\textbf{考试},更合适的心态是把它当成一次\textbf{相亲}






\subsubsection{结构化面试}


 结构化面试通常出现在公务员选调及一些国企事业单位的面试中。这类面试通常给出一个开放性的问题,提供一定的思考时间,让应试者展开回答。题型通常包括组织规划类、观点明晰类、人际交往类等。



针对结构化面试的应试策略可以分为两种:


一种我称之为\textbf{素材梳理式},适合绝大多数没有接触过行政话语的同学,因为按照往年各机构整理出的参考答案,可以发现,其实每一类题型都可以整理出对应的模板,里面嵌入的内容也有共性,比如组织规划类,经常出现事前要充分调研,事后要分析反馈,事中要维持秩序,做好应急预案等,那这些一个个通用的举措,就是我们要积累(背下来)的素材,再根据具体的题目情景进行细化的表述,就形成了一个各个环节紧密耦合,非常细致且全面的回答。这种方式需要通过大量的针对例题的实战练习,逐渐打磨出一套适合自己的答题模板,并积累内容素材和套话。练的特别熟悉之后,在正式面试时,通过快速反应、流畅表达和全面的内容输出给面试官留下深刻印象。这种方式的重点是练习时要掐时间脱稿,尽可能模拟面试现场情景,以克服紧张情绪,改掉不良口癖和动作,找到适合自己的表达方式。这样准备面试的同学,可以关注复旦大学的基层就业协会,它们每年都会在秋招期组建各个省的复旦官方选调群,复试期间里面会有复旦的同学们一起约线下对练,可以找一些同省考试的同学一起面对面练习,这样能听到对方给你的反馈意见,相互纠错。


% 另外一种我称之为\textbf{观点爆破式},适合平时就非常关注时政议题,有大量的观点累积,同时具有一定的思辨和表达能力的同学。这种方式需要我们突出重点,输出一个漂亮的观点,展示我们思考的深度。试想一下,在面你之前面试官可能都听了一天的模板,那些相同的套话和素材可能都听麻了,而这个时候你出场,从一个新颖的角度,细致阐述了一个有思考的论点,就会让面试官眼前一亮,脱颖而出。还用组织规划类的题型举例,这种策略的叙述格式类似于:虽然组织这个活动需要事前调研,准备,事后分析等等诸多环节,但是在这个活动里最关键/核心问题/最需要把握的环节其实是(比如说要做好紧急预案),因为这个活动具有某些鲜明的特点(比如大夏天在户外容易中暑啊,或者人员密集聚集容易踩踏,或者是线上直播容易有舆情之类的),曾经就有类似的活动因为没有注重这类隐患而出现了严重的问题(举例子),所以我们该吸取前车之鉴,做好以下工作(阐述举措)。所以梳理式是熟练有结构的罗列素材(类似BFS),爆破式是针对一个环节/观点进行深入阐述(类似DFS),这种准备的话需要同学们关注对应省份近一年来的重要议题,想一些自己的观点,并且记录下来,有新想法,或者在网上遇到了相关的案例,就把之前的记录更新,这样就能逐渐积累出自己的观点和案例库,增加现场面试时输出观点的机会。



另外一种我称之为\textbf{观点爆破式},适合平时关注时政议题、拥有大量观点积累并具有一定思辨和表达能力的同学。这种方式需要我们突出重点,输出一个漂亮的观点,展示我们思考的深度。
试想一下,在面你之前面试官可能都听了一天的模板,那些相同的套话和素材可能都听麻了,而这个时候你出场,从一个新颖的角度,细致阐述了一个有思考的论点,就会让面试官眼前一亮。
例如,回答组织规划类题型时,可以从活动的关键环节入手,指出最需要关注的地方(比如说虽然组织这个活动需要xxx,xxx等诸多环节,但是在这个活动里最关键/最核心问题/最需要把握的环节其实是要做好紧急预案),然后分析活动特点(如户外活动容易中暑、人员密集容易踩踏、线上直播易引发舆情等),并结合实例进行说明(比如曾经就有类似的活动因为没有注重这类隐患而出现了严重的问题),接下来再介绍举措并解释理由,最后几句话带过其他不重要的环节。
这类策略在准备上要求同学们关注报考省份近一年来的相关议题,形成自己的观点,并且记录下来。之后有新想法,或者在网上遇到了相关的案例,就不断更新之前的观点和案例库,以备面试时灵活应对。


总的来说,梳理式类似广度优先搜索(BFS),强调熟练的、完整的且结构化的素材罗列,可以短时间内速成;爆破式则类似深度优先搜索(DFS),着重深入阐述某一环节或观点,需要一些基础能力和日常积累。根据自身特点选择合适的应试策略,才能在面试中脱颖而出。

\subsubsection{无领导小组讨论/辩论赛}






无领导小组讨论或辩论赛形式的面试经常出现在国企以及部分省份的选调考试中,一些大厂的非技术岗位也会采用这种形式,有些甚至将两者结合,讨论过程中穿插辩论环节,最后还会要求进行总结。这类面试形式其实我们在校园中已有过一定的接触和体验。面对这样的求职面试,有以下几点经验可以参考:


\textbf{1.切记克制情绪。}这种面试不是为了吵赢其他的面试者,而是要把好的一面展示给旁听者,在和其他面试者们一起讨论。辩论的过程中肯定会有不同意见的碰撞,保持耐心倾听和礼貌表达,注意千万不要带着情绪,表现得过于强势,你一上头,旁观者就很下头,结果就是被挂。

\textbf{2.注意时间限制。}很多学长姐反馈的经验中都提到,如果自己的发言环节有时间限制,一定不要超时,尤其是在一些选调面试中。如果时间到了,就算没说完也要停下来,千万千万不要超时。

\textbf{3.争取有效输出。}
有效输出是指能够给面试官留下深刻好印象的表现。在无领导小组讨论中,虽然会有领导者(leader)、计时员(timekeeper)等角色,但校招面试中,面试官通常不会特别在意这些角色标签。一场面试下来,讨论的质量可能并不高。在这种情况下,有两个动作会特别出彩:

 1)总结:这是最有用的动作,总结也包括两部分,一个是面试的最后会要求选出一个人做总结汇报,这是一个备受关注的环节。如果能够争取到做总结的机会,并做好总结,会非常加分。即使无法担任最后的总结人,也要主动推举一个合适的人选,并给出充分的理由,避免在这个环节变成小透明。

另一个重要的部分是进行阶段性总结。当讨论变得混乱、大家提出许多不同角度的阐述时时,能够及时对内容进行概括和归纳,并向其他人确认,是非常有价值的。这种阶段性总结不仅可以使讨论更加清晰明了,还能引导大家进入下一个讨论环节。当面试官对讨论内容感到困惑时,他们往往会期待有人能够清晰地总结当前的讨论进展。这样的总结会让你很自然的成为推动讨论进程的人,并有效地掌控讨论的节奏。





 2)观点靠近业务:提出与应聘单位和岗位的业务内容相关的观点,能给面试官留下深刻印象。在面试前,多准备一些关于应聘单位和岗位的信息。在表达观点时,尽量结合这些要素,让人感受到你对这个单位或者这个岗位的充分了解和重视,会有意想不到的收获哦。


\subsubsection{主管面}
主管面一般都是单面,单面是最常见且基础的面试形式,可能是一对一的面试,也有可能是多对一的形式。你可能会面对多个业务骨干和HR,也可能只是与一个部门主管进行交流,但不论如何,面试者始终只有你一个人。


无论是大厂还是央国企,主管面都是整个面试过程中最关键的一环。面试官通常是主管或领导,他们通过这轮面试来了解新人,不仅决定候选人是否能够获得offer,还可能影响最终的岗位分配。

由于面试官的身份,他们对候选人的评估具有一定的影响力。根据学长姐们的经验,如果你在面试过程中与面试官交流愉快,即使最终因为人岗不匹配等原因,而未能获得该岗位的offer,也可能带来意想不到的收获。例如,面试官可能会将你的简历推荐给其他部门或不同的岗位、城市base。甚至能提供更好的机会让你去尝试,比如一些原本BG无法接触到的或无法通过简历筛选的岗位机会。因此,值得大家好好去把握。笔者根据自己的面试经验,以及采访过中学长姐们给到的一些建议,总结成以下几点经验,希望能给学弟妹们提供一些有效的参考:

\textbf{1.与其说是考试,其实更像相亲}



在这里将主管面比作相亲,是因为在这类面试中,面试官通常不会过多考察具体的知识和技能,而是更倾向于通过聊天的形式了解候选人。例如,他们会问一些类似“你做过的最有成就的一件事是什么”,或者“在面对压力时你通常怎么处理”这样的问题。通过这些问题,希望在对话中了解面试者是个什么样的人,以及了解一些你对于工作岗位的需求和看法,来判断一下你和岗位的匹配度。

在采访过程中,许多受访者提到,他们在面试时感觉自己回答得并不完美,但最终却收到了offer。入职了之后才发现,原来是当时的面试官觉得自己很特别。很多时候,这类面试的结果并不像考试那样单纯取决于回答的正确性,而更多取决于双方微妙的\textbf{情绪互动}。也不像考试一样是单向考察,而应该是\textbf{双向选择}。

一个理想的面试结果,是双方都聊得很愉快,并且互相留下了深刻的好印象。那么,如何才能实现这样的结果呢?实际上,这很大程度上取决于缘分。同样的表现,在不同的面试官眼中可能会产生截然不同的印象。因此,在面对这种不确定性时,我们需要做好自己能掌控的部分。这里笔者的建议,是一个在采访中各位学长姐频繁提到的词:\textbf{真诚}。


\textbf{2.宁可直白坦率,别做拙劣的骗子}

% 会有一些同学对自己的故事没有信心,灌水一些高大上的虚拟经历去应付问题,或者是为了迎合自己认为的面试官喜好,做了很多违心的回答。建议不要这样!

% 一是我们作为学生,被学校规训了这么多年,其实很难扮演一个高明的骗子,尤其在面试过程中,我们处在一个不对等的压力下,乱讲时那个飘忽的眼神,颤抖的双手,都很容易暴露。人家老油条一看就知道你在乱讲,一旦人家怀疑你在撒谎,那基本就无了。

% 二是就算糊弄成功,人家信了给了你offer,也未必是好事。这里的问题是我们为了获取一个offer,需不需要进行违心的\textbf{诉求表达}。我的建议是不要!因为这样收获的offer,大概率也是不那么符合心意的。没拿到这份工作不一定是坏事,糟糕的是进入了不适合自己的地方,还很难跳出来。比如面试官问你:能不能接受高强度的工作啊,怎么看待加班。你本来不能接受,但还违心的回答“加班,能让我成长,让我进步,我可以接受”,以为这样会更加分,最后offer给到了一个996的岗位,拿到了还不如不去。如果你看中的是高薪,直接表达“我不喜欢加班,但我需要更多的钱,如果给我很多的钱,我可以忍”会更好。

% 就像前面说的,主管面也是一个双向的选择,我们也在选择单位。就像在采访过程中,很多学长姐都觉得面试时,对面试官的感受,很大程度上就是之后一起工作时的感受,面试你的主管,今后很可能是你的部门主管,甚至更大一点的领导,是否和他们合得来,非常影响工作的满意度。在面试时真诚表达自己的感受和需求,也会收获真诚的反馈,甚至可能收获一些更有价值的建议和新机会。这很重要!可以帮助你提早进行筛选。找工作找的不是总数SUM,而是MAX。在过程中不用太担心自己会错过一两个的offer,这个世界这么大,我们的专业能报那么多岗位,只要坚持海投,即使快毕业了也有很多好机会,实在不行延毕也不是那么糟糕的事情。而过程中拿了很多差强人意的offer,反而会让你有了保底的懈怠,丧失斗志,结果反而错过了最好的机会。






有些同学在面对面试时,可能对自己的经历没有足够的信心,于是会编造一些“高大上”的虚假经历,或者为了迎合自己以为的面试官喜好,违心地回答一些问题。我的建议是,千万不要这样做!

首先,我们作为学生,经过了这么多年的规训,其实很难在面试时扮演一个“高明的骗子”。特别是在面试过程中,面对不对等的压力,撒谎时的飘忽眼神、颤抖双手,都可能瞬间暴露自己。那些经验丰富的面试官一眼就能看出你在乱讲,一旦他们对你的诚实产生怀疑,那基本就无了。

其次,即使侥幸通过了面试,拿到了offer,这也未必是件好事。关键在于,我们是否真的愿意为了拿到一份工作,而违心地进行\textbf{诉求表达}。我的建议是不要这样!因为通过这种方式拿到的offer,很可能并不符合你的真实需求。没有拿到这份工作未必是坏事,反而进入一个不适合自己的岗位,结果会更加痛苦,甚至难以脱身。

举个例子,如果面试官问你:“你能接受高强度工作和频繁加班吗?” 你本来内心是不能接受的,但为了拿到offer,违心地回答:“加班让我成长,我乐意接受。” 最后,或许你确实拿到了这个996的岗位,但工作后可能会发现自己难以承受这样的压力。这时,你可能会后悔当初没有诚实表达自己的需求。其实,如果你当时直接说:“我不喜欢加班,但我很需要多赚钱,我可以为了更多的薪资忍受一点加班。” 反而会让面试官对你的态度更加理解和尊重。

正如之前提到的,面试是一个双向选择的过程,我们不仅在争取offer,也在选择未来的工作环境。许多学长姐在面试中也提到,他们对面试官的感受很大程度上反映了之后的工作体验。面试你的主管,很可能就是你未来的直属领导,甚至是更大的领导。是否和这些人合得来,直接影响你未来的工作满意度。

因此,在面试时,真诚地表达你的感受和需求,往往会收到更真诚的反馈,甚至可能获得一些意想不到的建议或机会。这非常重要!它可以帮助你提前筛选出不合适的工作环境。找工作时,我们追求的不是拿到的offer数量(SUM),而是找到最适合自己的机会(MAX)。在这个过程中,不必担心错失一两个offer。这个世界很大,我们的专业能找到的岗位也很多。只要坚持海投,即使临近毕业,依然会有很多好机会。实在不行的话,延毕也并不是一个糟糕的选择。

反而,拿到太多差强人意的offer,可能会让你有了“保底”的懈怠,失去斗志,从而错过最佳机会。



\textbf{3.你是孤天里的鹤,不是写满字的纸}


% 有些同学会谦虚的在阐述自己经历和自我认知时会觉得自己的经历普普通通,没有什么特别厉害的东西可以拿的出来讲,会在网上找一些通用的模板来进行包装。但这样的效果未必好,相似的故事和话术,面试官可能都听麻了。那应该怎么办呢?

% 这里分享两个学长姐的案例:
% 其中一个是在主管面过程中被问到:请用几个词描述一下自己。这是个很常规的问题,一般的回答都会用“有好奇心,认真,有抗压能力”等词来夸夸自己。而这位学长当时回答的是:“幸运”,把视角从自我中抽离出来,阐述自己本不能走到今天,有幸一路走来遇到了很多人的善意和帮助。这个词当时给面试官留下了非常深刻的印象(入职后的反馈),最后当事人拿到理想了的offer

% 把视角从自我中抽离出来,放到了周围环境和人身上

% 第二个也是主管面,当事人被问到“你最有成就感的一段经历”,“认为自己具有的优势”,他说,我认为我的优势是本科不是科班,我能在本科毕业的时候即使找到未来的发展方向,并且立刻做出行动跨考,并完成,这是我很自豪的一件事。

% 描述词是幸运
% 非科班是优势



% 需要在我们在沟通过程中,展现出自己的闪光点,但这种展现的方式并不是罗列履历


% 本章将通过主管面的这一环节,分享笔者在单面过程中的经验和见解。


% 双向选择
% % 宁可清澈的愚蠢,也别做拙劣的骗子

% 总会有欣赏自己的那一部分
% 我们总会肯定自己有一些独特闪光,但是被优绩主义压抑着,不习惯于肯定这是个闪光点
% % 真诚,表达诉求,尤其在实习

% 真诚的

% 讲一个自己的故事,

% 双向选择


% 正向诠释自己的经历


有些同学在自我介绍或讲述自己经历时,可能会觉得自己的背景平平无奇,没有特别拿得出手的内容,于是倾向于在网上寻找一些通用模板来包装自己。然而,这样的方式未必能够产生好的效果。主管往往会对那些雷同的故事和话术感到厌倦。那么,应该怎么做才能脱颖而出呢?

这里分享学长姐的案例:

其中一个是在主管面试过程中被问到:“请用几个词描述一下自己。” 这是一个非常常见的问题,通常的回答往往会用“坚持、认真、抗压能力强”等词汇来夸夸自己。而这位学长当时的回答却是“幸运”。他讲述了自己原本可能不会走到今天,是因为一路上遇到了很多人的善意和帮助,才有了现在的机会。这一回答在面试官心中留下了非常深刻的印象(入职后的反馈),并且他最终获得了超过预期的offer。


通常我们在面试中描述自己的成就时,总是希望能强调自己的能力和优势,而“幸运”这个词似乎会弱化个人的努力和付出。然而,这种真实而谦逊的回答却让他的表达更加独特。它传递出一种对外界帮助的感激,以及对自身机遇的珍惜。这种真诚反而更能打动人心,效果也出乎意料地好。

另一个案例是当被问到“你对自己最自豪的事情是什么”时,有人这样回答:“我最自豪的事情是,我是非科班出身。”

通常非科班背景被认为是劣势,但如果你能够和许多科班出身的人竞争并走到主管面前,这本身就证明了你对行业的强烈热情、出色的学习能力以及执行力。为了追求自己的目标,你付出了比常人更多的努力,这反而是独特的优势。

很多被大众认为是劣势的特点或经历,其实都可以被反诠释,赋予新的价值。这不仅仅是一种包装技巧,更是心态上的转变。我们需要超越优绩主义的视角,去发掘那些曾经被忽视,但实际上自己真正认同、并能赋予我们独特价值的东西。

每个人内心都有一些让自己感到自豪的故事,然而我们往往因为害怕它们显得“不够高大上”或“不够体面”而选择回避这些内容。然而,那些看似微不足道的故事,可能正是你与众不同的所在。要勇敢的把这些相关的情绪、感受、细节挖掘出来,讲述出来,这个故事里面的情绪和细节填充的越丰富,就越容易构建起画面,引发共鸣。

不用害怕被否定,就像前文所述,主管面试更像是一场相亲。我们不需要用套路去取悦所有人,而是要找到真正欣赏我们的团队和工作环境。海投和面试不是为了“集邮”,而是为了找到那个最适合自己的机会。

