\chapter{择业指南}
%\chapter{计算机相关行业和岗位概述(以个人的观点篇为基本单元)}
% 在这一章,我们将介绍计算机专业能从事的各种行业和岗位,以及入行的渠道和经验。首先,我们会详细讲解互联网行业的各类岗位,因为互联网行业的流动性较高,不同公司间的工作内容也很相似,因此我们会以岗位为主线进行介绍,同时会简要提及一些公司的特点。

% 接下来,我们会按细分领域介绍其他行业的相关岗位,并与互联网行业的岗位进行比较,以便大家了解不同行业和对应岗位。
回顾自己的求职经历,我发现有许多冷门的优质单位以及一些看似与计算机学生无关的行业岗位,实际上都存在计算机专业进入的路径。然而,由于当时信息的局限和对专业的固有认知,这些机会都被错过了。因此,本章旨在为读者介绍计算机专业可以进入的各个行业和岗位,并分享前辈们对这些行业和岗位的见解与看法。内容将按行业划分,每个单元由个人观点的单篇构成,旨在帮助毕业生发掘更广阔的职业路径,并为选岗提供参考。这是我最希望完成的一部分,但目前内容还没怎么填充,现在仅有两篇完成。

如果您对某个行业有丰富的经验和独到的见解,并愿意成为编者在册子上撰写文章,欢迎通过邮箱 21210240339@m.fudan.edu.cn 或微信公众号“破蛋手册Beta”与我们联系。如果您在社交媒体上发布过相关的文章,并愿意将其转载到我们的册子中,也非常欢迎联系。我们会在筛选后收录,并注明原作者和原文链接。非常感谢您的支持!




%\section{互联网(算法/开发/运维/产品/项目管理/业务种类/组织架构)}
\section{互联网}
\subsection{互联网大厂核心部门之我见}

本篇内容整理自蒋雨宸学长的观点分享,已获得原作者授权,蒋学长是2024届南大计算机硕士毕业生,以下是他的微信号和知乎链接。

\textbf{微信号:}LIMBO\_42

\textbf{知乎主页: }\url{https://www.zhihu.com/people/luo-chen-96-77}
\subsubsection{背书程度排序}

无论是找实习还是正式工作,首要考虑的无非是两个维度:\textbf{背书和转正率}。而对于背书的重要性,我个人为之排了个序:\textbf{大厂核心部门>中厂核心部门>大厂非核心部门>中厂非核心部门>其他}

那么什么是互联网大厂呢?我个人认为只有BAT三家称之为大厂,只有他们三家业务很广,横跨多个领域。而其他公司例如:美团在本地生活,主要是在外卖上发力;快手短视频、pdd电商、小红书社区等,他们的业务都比较单一。相反,腾讯有社交,有支付,还有游戏;字节有短视频,咨询平台(头条懂车帝等等),也有短视频的衍生物如直播这种特殊的电商模式;阿里有阿里云、电商物流和金融平台。

下面我将逐一介绍每个大厂组织结构和核心部门(介绍顺序与企业地位无关)。

\subsubsection{腾讯}
\textbf{腾讯有六大事业群,分别如下:}
\begin{itemize}
    \item 企业发展事业群(CDG)
    
    主营业务为战略投资,咨询,市场公关等。这个事业群的岗位一般是一些职能类和市场类。
    
    \item 互动娱乐事业群(IEG)
    
    这是个游戏工作室,也是腾讯最值得去的事业群其一。作为腾讯的主要收入来源之一,IEG 被认为是公司内部极具吸引力的事业群。IEG 下辖有四大知名游戏工作室:天美、光子、魔方和北极光,每个工作室都有自己的特色和专业领域。
    
    在这些工作室中,还有进一步的细分团队,负责不同的项目和游戏。例如,天美工作室的 L1 团队负责开发广受欢迎的《王者荣耀》,而 Z1 团队则在寒假期间大力推广了《元梦之星》。IEG 的成功在于其能够孕育出市场上的爆款游戏,这不仅能为公司带来丰厚的利润,也能为团队成员提供极为优厚的待遇,如某些工作室成员能获得高达20个月工资的年终奖。

    然而,游戏行业的竞争同样激烈,项目的成功与否对团队的影响巨大。如果某个项目未能达到预期,可能会面临团队调整甚至裁员的风险,这一点在行业内并不罕见,如字节跳动的游戏部门“朝夕光年”所经历的情况。
    
    在游戏开发的过程中,不同部门的重要性也有所不同。引擎>客户端>后台/后端。
    \item 微信事业群(WXG):
    
    作为腾讯的核心业务之一,微信事业群(WXG)是公司最值得加入的部门之一。
    
    微信不仅是腾讯的流量命脉,也是其技术创新的重要基地。WXG 自主研发了众多内部技术解决方案,确保了微信服务的持续领先。视频号作为新兴功能,正迅速崛起,展现出与抖音竞争的潜力。同时,企业微信也在企业通讯和协作领域取得了稳健的发展。
    \item 技术工程事业群(TEG)
    
    作为公司的中台部门,技术工程事业群(TEG)扮演着基础设施建设者的角色。TEG 负责开发和维护数据库、网络和机器学习平台等核心技术架构,为整个公司提供坚实的技术支撑。
    \item 平台与内容事业群(PCG):
    
    平台与内容事业群(PCG)主要负责运营腾讯的老牌业务,包括腾讯QQ、腾讯视频和QQ浏览器等。这些业务凭借成熟的技术和稳定的用户基础,虽然增长空间有限,但仍是公司的重要组成部分。
    \item 云与智慧产业事业群(CSIG):
    
    云与智慧产业事业群(CSIG)以腾讯云为核心业务,致力于云计算和智慧产业解决方案的开发。尽管面临阿里云、华为云等强劲竞争对手,CSIG 仍在努力寻求突破。市场竞争激烈,且行业评价和业绩压力较大,CSIG 需要不断创新和优化服务,以提升市场竞争力。
\end{itemize}
\textbf{总结:}
\begin{itemize}
    \item 腾讯的两大核心事业群为互动娱乐事业群(IEG)和微信事业群(WXG),开发语言以前主要使用 C++,现在很多转go语言。
    \item 核心程度粗略排序:互动娱乐事业群(IEG) = 微信事业群(WXG) > 技术工程事业群(TEG) > 云与智慧产业事业群(CSIG) ≈ 平台与内容事业群(PCG)。
    \item 腾讯的关键部门包括天美、光子、优图实验室、AILAB等,这些部门在推动技术创新和产品研发方面发挥着关键作用。同时,微信公众号、视频号和企业微信作为公司的重要产品线,也是核心部门。
    \item 腾讯的岗位绝大部分在深圳,微信事业群(WXG)主要位于广州,而天美工作室在成都也设有分部。
    \item 工作强度来看,腾讯相对而言不是很卷。
    \item 腾讯官网投递需要选择事业群。
\end{itemize}

\subsubsection{字节}
字节跳动以其扁平化的组织架构而著称,这种架构带来了多条业务线,且相较于腾讯和阿里巴巴,其业务划分并不那么细致。以下是字节跳动的一些主要业务概览:

\begin{enumerate}
    \item 抖音业务群:涵盖抖音短视频平台、今日头条新闻客户端以及西瓜视频等多媒体内容服务。
    \item 飞书:一款集成了即时通讯、视频会议、日历、文档在线协作等功能的办公协作套件;
    \item 火山引擎:作为字节跳动的技术服务平台,火山引擎与阿里云和腾讯云竞争,提供云计算和人工智能服务。
    \item 朝夕光年:字节跳动的游戏部门,负责游戏开发和发行业务。
    \item TikTok:国际版的抖音,在全球范围内广受欢迎,特别是在年轻用户群体中。
    \item Data 部门:专注于搜索、广告推荐算法和数据服务,是字节跳动精准营销和个性化推荐的核心支持部门。
    \item 电商业务:字节跳动在电商领域的拓展,利用其庞大的用户基础和流量优势,涉足商品销售和带货服务。
    \item 其他业务:字节跳动还涉足了其他多个领域,例如小说阅读平台、汽车信息服务平台懂车帝等,持续拓展其业务范围和市场影响力。
\end{enumerate}

\textbf{总结:}
\begin{itemize}
    \item 核心部门:在公司中,AML(人工智能机器学习部门)占据核心地位,它不仅是最核心的机器学习和算法平台,还涵盖了AI基础设施建设。此外,抖音、TikTok的业务部门以及搜索、广告推荐平台(搜广推部门)也是公司的关键部门。
    \item 工作强度:在BAT(百度、阿里巴巴、腾讯)三大公司中,字节跳动以其高工作强度著称,被认为是“最卷”的公司。然而,这种高强度的工作节奏也伴随着公司快速的发展和广阔的职业前景。
    \item 工作城市:字节跳动的工作岗位主要集中在中国的一线城市,如深圳、北京和杭州等地,这些地方拥有较为集中的技术和业务团队。
\end{itemize}

\subsubsection{阿里系}
\begin{enumerate}
    \item 云智能集团:作为集团的核心技术支柱,云智能集团的业务涵盖阿里云智能、企业通讯协作平台钉钉、智能音箱天猫精灵以及专注于前沿科学研究的达摩院。
    \item 淘宝天猫商业集团:负责国内电商业务的核心板块,包括综合电商平台大淘宝(淘宝、天猫、阿里妈妈)、B2C零售事业群、社区团购业务淘菜菜、性价比电商平台淘特以及国内贸易平台CBU。
    \item 本地生活集团:以提升用户日常生活便利性为目标,主营业务包括地图导航服务高德和在线订餐平台饿了么。
    \item 菜鸟集团:专注于物流和供应链管理,为电商生态提供强有力的物流支持。
    \item 国际数字商业集团:拓展国际市场,业务包括东南亚电商平台Lazada、全球在线零售平台速卖通(AliExpress)和国际贸易平台ICBU。
    \item 大文娱集团:涵盖阿里巴巴的文化娱乐业务,旨在丰富用户的精神文化生活。
    \item 蚂蚁集团:以支付宝为核心,提供全面的金融服务,是集团金融科技的重要分支。
\end{enumerate}

\textbf{核心部门:}阿里云(ECS,PolarDB,各种存储包括OSS,SLS等等),淘天淘宝,阿里妈妈,淘宝首猜,蚂蚁,达摩院。

\textbf{工作地点:}杭州北京居多,少部分在上海

\subsubsection{中厂}

这里是我个人的一些刻板印象,不一定正确,欢迎指正:


\textbf{中厂第一梯队:}
\begin{itemize}
    \item 美团:公司主要工作地点集中在北京,而点评业务(大众点评)的少数岗位位于上海。核心部门包括到店业务和到家业务,分别对应大众点评和外卖服务。
    \item 拼多多:以其严格的工作时间而闻名,目前正全力发展其海外平台Temu,新入职员工大多会参与Temu项目,工作地点主要在上海。
    \item 快手:近期在电商领域进行了大规模裁员,但主站业务相对稳定。公司工作环境竞争激烈,工作地点以北京为主,同时在杭州和深圳也有部分职位。核心部门为社区科学线。
    \item 百度:尽管目前略显落寞,但公司正在积极开发大型模型,其成果尚待观察。百度的主要工作地点位于北京。
\end{itemize}


\textbf{中厂第二梯队:}

\begin{itemize}
    \item 小红书:发展势头强劲,实行大小周工作制,被个人看好,但正寻求盈利模式。与拼多多、得物并称为上海三大卷厂。
    \item 网易:专注于游戏领域,其技术栈与典型互联网公司有较大差异。
    \item 其他企业:包括b站、滴滴、携程、米哈游、蓝绿手机厂等。
    \item 大疆:工作环境竞争激烈,工作强度可与拼多多媲美。
    \item 微软、亚马逊:在国内的招聘活动已大幅减少。
    \item 英伟达:提供良好的工作生活平衡(WLB),在业界认可度高,尤其对于AI基础设施相关的职位,是一个值得考虑的选择。
    \item 米哈游:招聘活动很少,由于缺乏职位空缺(hc),导致招聘标准非常高,即使来自985高校的本硕毕业生也有简历筛选不通过的情况。
\end{itemize}

\textbf{卷度排序:}
根据网上的风评(不代表真实情况,实际情况可能因部门而异):

\begin{itemize}
    \item 拼多多的竞争激烈程度最高。
    \item 其次是小红书、大疆、得物。
    \item 字节跳动、快手、华为位列其后。
    \item 淘宝、天猫、蚂蚁、百度以及阿里巴巴和腾讯的一些部门紧随其后。
    \item 百度、阿里云、腾讯TEG、美团等位于排序的中部。
    \item 微软和英伟达的工作环境相对较轻松,位于排序的末尾。
\end{itemize}
\textbf{请注意,上述信息基于网络风评,具体情况可能因公司部门和岗位而异。}

% \subsection{算法}
% (qy0407更新)结构:1.算法行业总体介绍。2.行业中专业的需求和现状。3.简单介绍都有哪些类型的岗位(cv/nlp/搜广推/大模型)4.介绍不同类型的公司(大厂、中厂、小厂、国企等)
% \subsubsection{算法行业总体介绍}
% 算法是在互联网行业中用于高效解决特定问题或执行特定任务的一系列明确定义的计算步骤和方法,它能够在有限时间内产生预期的输出结果,是计算机科学中描述和解决问题的系统化策略。
% 随着技术的不断进步,算法已经渗透和扩展到各个领域的应用,如科技行业、教育行业、金融服务业、医疗健康业、制造业等等,将各个领域的应用和创新推动社会向更智能、更高校的方向发展,也在推动社会进步中发挥着关键作用。
% \subsubsection{算法行业中计算机专业的需求和现状:}
% 随着人工智能、物联网、区块链等新兴技术的不断发展,市场对计算机专业,尤其是算法行业的人才需求非常旺盛,需要大量的技术人才来推动和实现。目前对于算法行业的招聘岗位普遍有以下的招聘要求:
% 1.学历要求:硕士学历和博士学历居多,若部分岗位为本科要求,则工作经验需要1-3年左右;
% 2.技术要求:有较强的python编程基础,熟悉机器学习、深度学习等技术原理,且能够熟练应用。具体对应的技术要求要根据实际招聘岗位的业务需要和




% \subsection{开发}
% \subsection{运维/测试}
% \subsection{产品经理(刘济尘)}