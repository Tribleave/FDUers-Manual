\section{金融业}


% \subsection{金融监管和基础设施}


% 如果把金融交易行为比作马路上来来往往的车流,那么金融监管就相当于这条路上的交通警察。政府通过特定的机构对金融交易行为主体进行限制或规定。我国的金融监管体系以一行(中国人民银行及其管理的国家外汇管理局)、一会(中国证券监督管理委员会)和一局(原银保监会,现为金融监督管理总局)及其分支机构为核心。

% 中国人民银行主要负责制定和执行货币政策,维护金融稳定,提供金融服务。中国证券监督管理委员会的职责是统一监督管理全国证券期货市场,维护市场秩序。金融监督管理总局则对除证券业之外的金融业实行统一监督管理。

% 自2023年机构改革后,这些单位岗位都变成了公务员。三个正部级单位在全国各地设有众多分支机构,报考这些职位需要通过国家公务员考试。考试报名时间一般在每年的十月中下旬到十一月初。

% 对于计算机专业的毕业生,可以关注每年各机构发布的招录计划。有些年份会招收具有计算机背景的考生。例如,中国证券监督管理委员会的科技监管司就曾招录计算机专业背景的考生。总部机关实际上相当于部委,适合有政治理想且经济条件宽裕的同学报考。有关备考策略、薪酬待遇和职业发展路径等内容将在后续的“公务员”章节中介绍。


% 要保障金融活动的正常运行,除了需要有金融“交通警察”之外,还需要确保每一笔交易都有唯一的“车牌号”(登记),并设置“收费站”和“检查站”(结算)。同时,金融基础设施还包括“红绿灯”、“沥青路面”、“指示牌”和“服务区”等等(系统维护)。金融基础设施为金融业活动搭建平台,提供金融运行的硬件设施和制度安排,同时起到监管支撑的作用,由金融监管单位领导。

% 国内金融基础设施的组织形态大致分为三类:

% 1.公司制
% 2.事业单位
% 3.会员制交易所和行业自律组织(如交易商协会)
% 从业务性质上,金融基础设施主要分为五类:

% 支付系统(PS):如中国人民银行清算总中心运营的大额支付系统、银联跨行支付系统。

% 中央托管机构(CSD)和证券结算机构(SSS):如中央结算、中国结算、上海清算所。

% 中央对手方(CCP):如上海清算所。

% % 交易报告库(TR):如中国期货市场监控中心和中证报价。
% 交易报告库(TR)

 
% 这些工作单位的优势在于,相较于市场机构来说,具有监管支撑属性,行业地位比较高,业务有一定垄断属性,工作强度一般不算大,很稳定,薪资也有一定竞争力,从wlb的角度来看性价比较高。劣势在于薪资的成长相较于市场机构没有那么快,而且业务面较窄(对于市场来说),跳槽空间有限。每年从9月份开始陆续招聘,报名和考试集中在每年11月-次年2月。基本上每年只招聘一次。对于这类基础设施来说,系统建设变得越来越重要,近些年来招聘IT背景的应届生有扩招趋势,不光是技术部门招聘,也可能是一些业务部门对接技术的岗位。可以关



\subsection{银行选岗二三事}



许多CSer在找工作前可能听说过,计算机专业毕业后可以去银行。一般来说,大家对银行的印象是,相较于互联网大厂,银行的薪资可能略低,但工作强度较小,且相对稳定。然而,对于自己究竟该进入哪些银行机构,适合选择哪些岗位,往往并没有太多的了解。希望本篇能够为回答这两个问题提供一点参考。

\subsubsection{银行种类及组织架构}
商业银行是金融行业的基础,但当前金融行业面临整顿,银行贷款利率降低,且有些面临坏账风险,国有银行员工还面临降薪压力。短期内银行业面临诸多挑战。

在银行种类上,主要有如下分类:

1.宏观货币政策机构:央行(中国人民银行)。

2.政策性银行:包括国家开发银行(2015年被定位为开发性金融机构)、中国农业发展银行和中国进出口银行。
这类银行是以贯彻政府的经济政策为目标,在特定领域开展金融业务的不以营利为目的的专业性金融机构,不对大众提供储蓄业务。

3.国有银行:工商银行、农业银行、中国银行、建设银行、交通银行和邮政储蓄银行。
是指由国家(通常借由财政部、中央汇金公司出资)直接管控的大型商业银行。

4.股份制银行:如招商银行、光大银行等。
一般有企业法人持股

5.地方城商行:如北京银行,宁波银行等。
大股东一般是地方政府、国有企业或大型公司

6.外资银行:如渣打,恒生等。


以及农商行信用合作社等机构。

对于应届生来说,可以先有一个粗略的理解:央行、政策性银行、国有银行、股份制银行从左到右逐渐偏向市场化企业,而从右到左则逐渐更像机关。其中,人民银行是国务院的组成部门;政策性银行和除邮政储蓄银行之外的国有银行均为金融央企,总部除了交行在上海之外,其余全在北京;其中邮政储蓄银行是中国邮政集团的全资子公司。通常越往左边,行业地位更高,稳定性更好,但薪资成长比较平稳;越往右边,薪资涨幅空间越大,相应地也要承担更多市场化的竞争。





在组织结构上,银行通常采用总行(及总行直属机构和子公司)-省分行-支行的架构。我们日常去办理业务的网点通常是支行。在这些岗位中,部分省分行(尤其是政策性银行的省分行)和一些总行直属机构/子公司(如理财子公司)及总行机关都是很值得考虑的工作单位。

对于应届生来说,总行机关通常在行内地位、发展平台和稳定性方面表现更佳,因此竞争更加激烈,门槛也更高。政策性银行和国有银行总行因在各方面表现均衡(六边形战士),每年都是报考的热门。其中,普遍认为建设银行总部的待遇最优(在北大BBS上被戏称为“剑宗”)。政策性银行里,农发行总行待遇也很不错,应届生同学可以多关注。

然而,国有行总行未必对所有应届生都是最合适的选择。一些省分行、子公司和直属机构以及股份制银行,由于更接近业务前线,可以提供更高的薪资水平和更广的跳槽空间。同时,性价比也是一个重要考量因素。比如,某政策性银行的省分行在薪资上并不比总行低多少,但所在城市的生活成本却远低于北京。此外,兴趣爱好也需考虑。有些同学热爱编程,志在技术深耕,那总行机关甚至整个银行业可能都不适合他们。在选择时,不要盲目跟风。根据自身情况和需求进行取舍,才是更明智的做法。

\subsubsection{岗位种类}

近年来,随着金融科技的发展,银行业迎来了一波信息化改革,因此提供了大量计算机专业毕业生可以报的岗位。这些岗位主要分为两类:

1.机关类型单位:如总行金融科技部或其他总行部门。这类岗位的职责类似于产品经理或项目管理,主要负责对接业务部门和技术部门,沟通和翻译工作较多。HC可能集中在金融科技部,也可能在业务部门或中后台部门单独招收具有技术背景的毕业生。

2.IT技术类型单位:如总行直属的数据中心、软件开发中心或金科子公司等。这类单位主要承担全行日常运营维护、软件开发和数据分析等任务,岗位性质更接近程序员。

对于总行机关类型的单位,得益于金融科技的火热,除了中后台部门外,总行大量前台业务部门也允许计算机类专业应届生报考,以建设银行总行2024届校招为例(如图\ref{岗位表}所示),40多个总行部门校招,其中有30多个部门计算机类专业可以报考,横跨前中后台,其中包括核心业务部门(如金融市场部,公司业务部等),核心职能部门(如党委办公室等),以及一些很好的后台支撑部门(安全保卫部等)。进入之后,总行不同部门之间会有流动性,有机会转到其他部门,做与专业背景相关性不大的金融业务工作,更何况业务部门本身也招收计算机专业的毕业生,部门内部的分工更非一成不变。因此,这类岗位非常适合那些不希望未来工作主要以编写代码为主的同学,类似于公务员的工作性质,也需要撰写大量的公文材料。具体内容可参考北大BBS某热门帖子
~\url{https://bbs.pku.edu.cn/v2/post-read.php?bid=99\&threadid=18321878}。

\begin{figure}[htbp]
    \centering
    \includegraphics[width=\linewidth]{img/2024CCB.pdf}
    \caption{24年建总岗位需求表}
    \label{岗位表}
\end{figure}



对于IT类技术单位,如交通银行软件开发中心、数据中心,邮储银行数据中心,以及建信金科等,通常是总行直属机构(条线)或子公司。这些单位近年来为应届计算机科学专业毕业生提供了大量岗位,包括开发、运维、产品管理、AI算法等,主要承担银行的IT研发和日常运维任务。相比互联网行业,这类单位的工作强度相对较小,但仍然存在一定的工作压力。例如,数据中心的运维岗位需要定期值班,工作并没有想象中那么轻松。在薪资方面,起薪和涨幅通常也低于互联网行业。但是很稳定,且有一些地点在二线城市的单位综合性价比很高(如邮储行在合肥的数据中心)。适合那些追求工作稳定、厌恶高强度的同学。



\subsubsection{报考注意事项}


报考方面,中国人民银行作为国务院组成部门,2023年机构改革之后,需要通过公务员国考渠道进行报考。而其他各银行则会在每年秋季开展校招工作,同学可以通过各家银行的校园招聘网站进行报名。需要注意的是,多家银行的招聘对英语能力有一定要求。例如,建设银行总部要求报考者通过大学英语六级考试(成绩不低于425分)或其他同等水平的英语考试成绩;进出口银行总部则要求六级成绩达到450分及以上,或其他同等水平的英语考试成绩。因此,想要报考的同学需提前了解往年要求,并准备好英语成绩,以免失去报考资格。


考试通常包括笔试和后续的面试,更多地采用类似体制内的考察方式。笔试主要涵盖行政能力测试(行测)、专业知识、英语以及单位的特色文化知识(如标语、理念、发展历史等)。专业知识部分包括金融知识和信息技术。有些银行在笔试时会根据专业和岗位将考试分为两张试卷(金融类/信息科技类),如中国银行。

信息技术类的考察内容虽然较为基础但涉及面广,需要广泛备考。即使是信息科技类的考试,也会涉及少量的金融知识,建议在备考时粗略了解一些基本的金融常识。具体笔面试可以参考第二章节,一般银行面试问的技术问题都比较少(甚至没有),即使是技术单位技术岗也相对较少,通常也只是验证一下对简历项目的了解程度。 